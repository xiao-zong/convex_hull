
\documentclass[12pt]{article}
\usepackage{amsfonts}
\usepackage{textalpha} % Enable Greek text

\usepackage{bm}
\usepackage{enumerate} 
\usepackage{amssymb,amsmath}
\usepackage{caption}
\usepackage{subcaption}
\usepackage{accents} 
\usepackage{dsfont}
\usepackage{stackengine}
\let\Horig\H

\usepackage{tikz}
\usetikzlibrary{automata,topaths}
\usetikzlibrary{shapes}
\usetikzlibrary{plotmarks}
 
% \usepackage{wasysym} 
\usepackage{float} 
\usepackage{textcomp} 
\usepackage{siunitx}
\sisetup{output-exponent-marker=\ensuremath{\mathrm{e}}}

\usepackage{color,cite} 
\definecolor{lightblue}{rgb}{0,0.2,0.5}
\usepackage[colorlinks=true, urlcolor=lightblue,linkcolor=lightblue, citecolor=lightblue]{hyperref}
\usepackage{ucs}
%% For typesetting code listings                                                

% \usepackage{listings}
\usepackage{listingsutf8}
\lstdefinelanguage{Sage}[]{Python}
{morekeywords={False,sage,True},sensitive=true}

\lstset{
 mathescape = false,
 escapechar = {$},
 basicstyle = \ttfamily,
 extendedchars=false,
 inputencoding=utf8,
%  frame=none,
%  showtabs=False,
%  showspaces=False,
%  showstringspaces=False,
  commentstyle={\ttfamily\color{dgreencolor}},
  keywordstyle={\ttfamily\color{dbluecolor}\bfseries},
  stringstyle={\ttfamily\color{dgraycolor}\bfseries},
  language=Sage,
  basicstyle={\fontsize{10pt}{10pt}\ttfamily},
  aboveskip=0.3em,
  belowskip=0.1em,
  numbers=none, % left,
  numberstyle=\footnotesize,
  breaklines=true,                % sets automatic line breaking 
}

\definecolor{dblackcolor}{rgb}{0.0,0.0,0.0}
\definecolor{dbluecolor}{rgb}{0.01,0.02,0.7}
\definecolor{dgreencolor}{rgb}{0.2,0.4,0.0}
\definecolor{dgraycolor}{rgb}{0.30,0.3,0.30}
\newcommand{\dblue}{\color{dbluecolor}\bf}
\newcommand{\dred}{\color{dredcolor}\bf}
\newcommand{\dblack}{\color{dblackcolor}\bf}

\usepackage{graphicx}
\usepackage{flushend,cuted}
\usepackage{bm}
\usepackage{tabularx}
%\usepackage{color}
\usepackage{indentfirst}
\usepackage{amssymb}
\usepackage{xparse}
\usepackage{tikz}
\usepackage{mdwlist}
\usepackage{tkz-graph}

\DeclareMathAlphabet{\eufrak}{U}{}{}{} 
\SetMathAlphabet\eufrak{normal}{U}{euf}{m}{n}
\SetMathAlphabet\eufrak{bold}{U}{euf}{b}{n}

% \usepackage{eucal}

% \usepackage{epsfig,latexsym}

% \usepackage{graphicx,amsmath,amssymb,latexsym,psfrag}

% \usepackage[polish]{babel}

\newtheorem{assumption}{Assumption}[section]

% \usepackage{graphics,graphicx,amsmath}

\oddsidemargin=0cm \textwidth=16.5cm \textheight=23cm
\topmargin=-1.5cm
\newcommand{\R}{\mathbb{R}}
\newcommand{\V}{\mathbb{V}}
\newcommand{\T}{\mathbb{T}}
\newcommand{\C}{\mathbb{C}}
\newcommand{\E}{\mathbb{E}}
\newcommand{\IP}{\mathbb{P}}
%\newcommand{\bone}{\bone}
\newcommand{\bone}{{\bf 1}}
% \newcommand{\E}{\mathrm{E}}

\newcommand{\supp}{\mathrm{supp}}

%% \renewcommand{\le}{\leqslant}
%% \renewcommand{\leq}{\leqslant}

%% \renewcommand{\ge}{\geqslant}
%% \renewcommand{\geq}{\geqslant}

\newcommand{\conv}{\mathrm{conv}}
\newcommand{\card}{\mathrm{card}}
\newcommand{\grad}{\mathrm{grad}}
\newcommand{\N}{\mathbb{N}}
\newcommand{\inte}{\mathbb{N}}
\newcommand{\bbf}{{\mathbf{f}}}
\newcommand{\bT}{\mathbb{T}}
\newcommand{\TP}{\widetilde{P}}
\newcommand{\tgi}{t\rightarrow \infty}
\newcommand{\ngi}{n\rightarrow \infty}
\newcommand{\algi}{\alpha \rightarrow \infty}
\newcommand{\xgi}{x\rightarrow \infty}
\newcommand{\oJ}{\overline{J}}
\newcommand{\og}{\overline{\gamma}}
\newcommand{\oL}{\overline{\Lambda}}
\newcommand{\EPSI}{\varepsilon}

% To define \widebar
\makeatletter
\newcommand*\rel@kern[1]{\kern#1\dimexpr\macc@kerna}
\newcommand*\widebar[1]{%
  \begingroup
  \def\mathaccent##1##2{%
    \rel@kern{0.8}%
    \overline{\rel@kern{-0.8}\macc@nucleus\rel@kern{0.2}}%
    \rel@kern{-0.2}%
  }%
  \macc@depth\@ne
  \let\math@bgroup\@empty \let\math@egroup\macc@set@skewchar
  \mathsurround\z@ \frozen@everymath{\mathgroup\macc@group\relax}%
  \macc@set@skewchar\relax
  \let\mathaccentV\macc@nested@a
  \macc@nested@a\relax111{#1}%
  \endgroup
}
\makeatother

\makeatletter
\DeclareRobustCommand\widecheck[1]{{\mathpalette\@widecheck{#1}}}
\def\@widecheck#1#2{%
    \setbox\z@\hbox{\m@th$#1#2$}%
    \setbox\tw@\hbox{\m@th$#1%
       \widehat{%
          \vrule\@width\z@\@height\ht\z@
          \vrule\@height\z@\@width\wd\z@}$}%
    \dp\tw@-\ht\z@
    \@tempdima\ht\z@ \advance\@tempdima2\ht\tw@ \divide\@tempdima\thr@@
    \setbox\tw@\hbox{%
       \raise\@tempdima\hbox{\scalebox{1}[-1]{\lower\@tempdima\box
\tw@}}}%
    {\ooalign{\box\tw@ \cr \box\z@}}}
\makeatother

\newcommand{\disc}{\mathrm{disc}}
\newcommand{\bZ}{\bold{Z}}
\newcommand{\bz}{\bold{z}}
\newcommand{\dtv}{{d_{\rm TV}}}
\newcommand{\dk}{{d_{\rm K}}}
\newcommand{\dw}{{d_{\rm W}}}


\newtheorem{prop}{Proposition}[section]
\newtheorem{lemma}[prop]{Lemma}
\newtheorem{definition}[prop]{Definition}
\newtheorem{corollary}[prop]{Corollary}
\newtheorem{thm}[prop]{Theorem}
\newtheorem{remark}[prop]{Remark}
\newtheorem{example}[prop]{Example}

%\def\bone{\vmathbb{1}}
\def\bp{\noindent{\it Proof.}\ }
\def\ep{\hfill $\Box$}
\newcommand{\bt}{\mathbf{t}}
\def\({\left(}
\def\){\right)}
% \theoremstyle{definition}
% \newtheorem{definition}{Definicja}[section]
\newcommand{\cov}{\mathrm{Cov}}
\def\[{\left[}
\def\]{\right]}
\def\real{{\mathord{\mathbb R}}}
\def\N{{\mathord{\mathbb N}}}
\def\Dom{\mathrm{Dom}}
\def\Var{\mathrm{Var}}
% \newcommand{\p}{\mathbb{P}}
\newcommand{\pr}{\mathbb{P}}
\def\P{\mathbb{P}}
% \newcommand{\R}{\mathbb{R}}
%\newcommand{\N}{\mathbb{N}}
\newcommand{\Z}{\mathbb{Z}}
% \newcommand{\C}{\mathbb{C}}
% \newcommand{\E}{\mathbb{E}}
\newcommand{\re}{\mathbb{e}}

\newenvironment{Proof}{\removelastskip\par\medskip
\noindent{\em Proof.} \rm}{\penalty-20\null\hfill$\square$\par\medbreak}

\newenvironment{Proofx}{\removelastskip\par\medskip
\noindent{\em Proof.} \rm}{\par}

\newenvironment{Proofy}{\removelastskip\par\medskip
\noindent{\em Proof} \rm}{\penalty-20\null\hfill$\square$\par\medbreak}

\allowdisplaybreaks

\numberwithin{equation}{section}

\usepackage{fvextra}

% \usepackage{refcheck}
%%%%%%%%%%%%%%%%%%%%%%%%%%%%%% for drawing pictures by tikz
\GraphInit[vstyle = Shade]
\usetikzlibrary[intersections,
positioning,
petri,
backgrounds,
fit,
decorations.pathmorphing,
arrows,
arrows.meta,
bending,
calc,
intersections,
through,
backgrounds,
shapes.geometric,
quotes,
matrix,
trees,
shapes.symbols,
graphs,
math,
patterns,
external,
scopes,
matrix,
lindenmayersystems,
shapes.callouts,
shapes.misc,
angles,
shapes.arrows,
shadings]
%%%%%%%%%%%%%%%%%%%%%%%%%%%%%%%%%%%%%%%%

\makeatletter
\lst@InputCatcodes
\def\lst@DefEC{%
 \lst@CCECUse \lst@ProcessLetter
  ^^80^^81^^82^^83^^84^^85^^86^^87^^88^^89^^8a^^8b^^8c^^8d^^8e^^8f%
  ^^90^^91^^92^^93^^94^^95^^96^^97^^98^^99^^9a^^9b^^9c^^9d^^9e^^9f%
  ^^a0^^a1^^a2^^a3^^a4^^a5^^a6^^a7^^a8^^a9^^aa^^ab^^ac^^ad^^ae^^af%
  ^^b0^^b1^^b2^^b3^^b4^^b5^^b6^^b7^^b8^^b9^^ba^^bb^^bc^^bd^^be^^bf%
  ^^c0^^c1^^c2^^c3^^c4^^c5^^c6^^c7^^c8^^c9^^ca^^cb^^cc^^cd^^ce^^cf%
  ^^d0^^d1^^d2^^d3^^d4^^d5^^d6^^d7^^d8^^d9^^da^^db^^dc^^dd^^de^^df%
  ^^e0^^e1^^e2^^e3^^e4^^e5^^e6^^e7^^e8^^e9^^ea^^eb^^ec^^ed^^ee^^ef%
  ^^f0^^f1^^f2^^f3^^f4^^f5^^f6^^f7^^f8^^f9^^fa^^fb^^fc^^fd^^fe^^ff%
  ^^^^03b6% <--- for ζ
  ^^^^03b1^^^^03b2^^^^03b3%
  ^^00}
\lst@RestoreCatcodes
\makeatother

\begin{document}
\title{
\huge
 Phase transitions of subgraph counts in random-connection model with endpoints
} 

\author{
  Qingwei Liu\footnote{\href{mailto:qingwei.liu@ntu.edu.sg}{qingwei.liu@ntu.edu.sg}}
  \qquad
      Nicolas Privault\footnote{
\href{mailto:nprivault@ntu.edu.sg}{nprivault@ntu.edu.sg}
}
  \\
\small
Division of Mathematical Sciences
\\
\small
School of Physical and Mathematical Sciences
\\
\small
Nanyang Technological University
\\
\small
21 Nanyang Link, Singapore 637371
}

\maketitle

\vspace{-0.5cm}

\begin{abstract} 

\end{abstract}
\noindent\emph{Keywords}:~
Random-connection model, 
subgraph count,
Phase transition,
cumulant method,
Poisson point process,
random graphs.

\noindent 
{\em Mathematics Subject Classification:} 
60F05, % Central limit and other weak theorems
60D05, % Geometric probability and stochastic geometry
05C80, % Random graphs (graph-theoretic aspects)
60G55. % Point processes (e.g., Poisson, Cox, Hawkes processes)
% 60G57	Random measures
%60B10. % Convergence of probability measures
 
\baselineskip0.7cm

\section{Introduction}
In a random geometric graph, the vertex set is a point process and each pair of vertices would be connected by an edge if and only if the distance between the two vertices are smaller than a fixed threhold. 
A random-connection model (RCM) is a random graph whose vertex set is a point process $\xi$ on $\R^d$, $d\ge1$, and each pair of vertices are randomly connected with a location-dependent probability, which is usually given by a connection function $H:\R^d\times \R^d\to[0,1]$. In particular, when the connection function $H$ is a function of the distance between pairs of points of $\xi$, i.e. $H(x,y):=\phi(\|x-y\|)$ for some measurable function $\phi:\R_+\to[0,1]$, the resulting graph is also known as a {\it soft} random geometric graph, c.f. \cite{penrose16,LNS21}. 

The random-connection model can be regarded as a generalisation of several classic random graph. For instance, when the connection function $H$ is given by an indicator function, of the form $\bone_{\{\|x-y\|\le r\}}$, for some $r>0$, the random-connection model becomes a random geometric graph, c.f. the monograph \cite{penrosebk}. When taking the underlying point process $\xi$ to be a Binomial point process $\xi_n$, meanwhile the connection function $H(x,y)\equiv p$ for some $p\in(0,1)$, the resulting graph is the Erd{\H o}s-R\'enyi random graph, c.f. \cite{gilbert59,ER}. Another special case of random-connection model is so-called graphon based random graph, c.f. \cite{zhang22,bhattacharya23}, when the underlying point process is a Binomial point process $\xi_n$ on $[0,1]$ with a connection function, also known as a graphon, $W:[0,1]\times[0,1]\to[0,1]$.

In a recent preprint \cite{LiuPrivault23}, normal approximation for the subgraph counts in the random-connection model with convergence rates under the Kolmogorov distance is established through an ingenious diagram language and combinatorial arguments. Through the analysis of cumulants under different limiting regimes, asymptotic normality of the subgraph counts is discovered in \cite{LiuPrivault23} for connected graph $G$ in a dilute regime and for $G$ being a tree in certain sparse regime. 

In the present paper, we investigate the subgraph counts in random-connection model with and without fixed endpoints and give a full consideration for phase transition phenomenon corresponding to the asymptotic orders of cumulants. 

The randon-connection model considered in this paper is defined as follows. For $\lambda > 0$, let $\Xi_\lambda$ be a Poisson point process on $\R^d$, $d\geq 1$, with diffuse $\sigma$-finite intensity measure $\Lambda_\lambda$. According to \cite[Corollary~6.5]{LastPenrose17}, $\Xi_\lambda$ can be almost surely written as 
$$\Xi_\lambda=\sum_{n=1}^\tau\delta_{X_i},$$
where $\tau$ is a $\N\cup\{\infty\}-$valued random variable, $\delta_x$ denotes the Dirac measure at $x\in\R^d$, and $X_1,X_2,\dots$ are random elements in $\R^d$. Let $m\in\N$, $y_1,\dots,y_m\in\R^d$, and $[m]:=\{1,\dots,m\}$. We define a point process $\Xi_{\lambda,m}$ on $\R^d$ by 
$$\Xi_{\lambda,m}:=\sum_{n=1}^\tau\delta_{X_i}+\sum_{i\in[m]}\delta_{X_{-i}},$$
where $X_{-i}:=y_i$ for $i\in [m]$. Note $\Xi_{\lambda,m}=\Xi_\lambda$ if $m=0$. Let $H:\R^d\times \R^d\to [0,1]$ be a symmetric measurable function, and also denote $H_\lambda:=c_\lambda\cdot H$, for some $c_\lambda\in(0,1)$. The random-connection model we considered in the present paper is a random graph with vertex set $\Xi_{\lambda,m}$, such that any two distinct vertices $x,y\in\Xi_{\lambda,m}$ are independently connected by an edge with probability $H_\lambda(x,y)$. The random-connection model is denoted by $G_{H_\lambda}(\Xi_{\lambda,m})$, and we say that $H_\lambda$ is the connection function of random-connection model $G_{H_\lambda}(\Xi_{\lambda,m})$.

In this paper, we mainly consider the counts $N_G$ of subgraph in the random-connection model $G_{H_\lambda}(\Xi_{\lambda,m})$, which is isomorphic to a certain connected graph $G$ and meanwhile it contains some fixed endpoints as its vertices. The formal mathemcatical definition of $N_G$ is given in Section~\ref{prelim}. We next present some of our results on the random-connection model $G_{H_\lambda}(\Xi_{\lambda,m})$ when the intensity measure $\Lambda_\lambda$ is of the form $$\Lambda_\lambda=\lambda\cdot\mu,$$
for some diffuse $\sigma$-finite measure $\mu$ on $\R^d$. 

% Let $G$ be a connected graph with $r+m$ vertices, $r\ge2$. 
% For the case when $m=0$, suppose that there exists a constant $\varepsilon>0$ such that 
% \begin{equation}
%   \int_{(\R^d)^{2r-1}}\left[\prod_{1\le i<j\le 2r-1}H(x_i,x_j)\right]\mu\{\mathrm{d}x_1\}\cdots\mu\{\mathrm{d}x_{2r-1}\}\ge \varepsilon^{2r-1}.
% \end{equation}

\begin{thm}\label{thm-1}
  Let $G$ be a connected $K_2$-balanced graph with $r+1$ vertices, $r\ge2$. Suppose there exists a constant $\varepsilon>0$ such that 
  \begin{equation}
    \int_{(\R^d)^{2r-1}}\left[\prod_{1\le i<j\le 2r}H(x_i,x_j)\right]\mu\{\mathrm{d}x_1\}\cdots\mu\{\mathrm{d}x_{2r-1}\}\ge \varepsilon^{2r-1},
  \end{equation}
  where $x_{2r}:=y_1$. Let $N_G$ be the count of subgraph on $G_{H_\lambda}(\Xi_{\lambda,1})$ which contains one fixed endpoint $y_1\in\R^d$. Taking $c_\lambda\to0$ as $\lambda\to\infty$, we have
  \begin{eqnarray}
    \kappa_n\big(N_G\big)=\left\{\begin{array}{ll}
      \Theta\big(\lambda^{(r-1)n+1} c_\lambda^{(e(G)-1)n+1}\big) & \ \mbox{if }\ \lambda c_\lambda^{[e(G)-1]/(r-1)}\gg1,
      \medskip
      \\
      \displaystyle
\Theta\big(\lambda^{r}c_\lambda^{e(G)}\big) & \ \mbox{if }\ \lambda c_\lambda^{[e(G)-1]/(r-1)}\ll1,
    \medskip
    \\
    \displaystyle
    \Theta(\lambda c_\lambda) & \ \mbox{if }\ \lambda c_\lambda^{[e(G)-1]/(r-1)}\asymp1.
    \end{array}
    \right.
  \end{eqnarray}
  where $\kappa_n(N_G)$ stands for the $n$-th cumulant of $N_G$, for all $n\ge 1$, and $e(G)$ is the number of edges of graph $G$.
\end{thm}

\begin{thm}\label{thm-2}
  Let $G$ be a strongly balanced and connected graph with $r$ vertices, $r\ge2$. Suppose there exists a constant $\varepsilon>0$ such that 
  \begin{equation}
    \int_{(\R^d)^{2r-1}}\left[\prod_{1\le i<j\le 2r-1}H(x_i,x_j)\right]\mu\{\mathrm{d}x_1\}\cdots\mu\{\mathrm{d}x_{2r-1}\}\ge \varepsilon^{2r-1}.
  \end{equation}
 Let $N_G$ be the count of subgraph on $G_{H_\lambda}(\Xi_{\lambda})$. Taking $c_\lambda\to0$ as $\lambda\to\infty$, we have
  \begin{eqnarray}
    \kappa_n\big(N_G\big)=\left\{\begin{array}{ll}
      \Theta\big(\lambda^{(r-1)n+1} c_\lambda^{ne(G)}\big) & \ \mbox{if }\ \lambda c_\lambda^{e(G)/(r-1)}\gg1,
      \medskip
      \\
      \displaystyle
\Theta\big(\lambda^{r}c_\lambda^{e(G)}\big) & \ \mbox{if }\ \lambda c_\lambda^{e(G)/(r-1)}\ll1,
    \medskip
    \\
    \displaystyle
    \Theta(\lambda) & \ \mbox{if }\ \lambda c_\lambda^{e(G)/(r-1)}\asymp1.
    \end{array}
    \right.
  \end{eqnarray}
  where $\kappa_n(N_G)$ stands for the $n$-th cumulant of $N_G$, for all $n\ge 1$, and $e(G)$ is the number of edges of graph $G$.
\end{thm}



 

\section{Preliminaries and notations}\label{prelim}
\subsection*{Set partition and Diagrams}
\noindent 

\subsection*{Graph theory}
\noindent
In what follows, given any two graphs $G_1,G_2$
we write $G_1\simeq G_2$ when $G_1$ is isomorphic to $G_2$. 
We also let $v(H):=|V(H)|$ and $e(H):=|E(H)|$ be the number of vertices and the number of edges of a graph $H$.
Throughout this note, $G$ is assumed to be a connected graph with $ r = v(G)$ vertices. %  and $e(G)=k$.
We denote by 
$$
{\rm CNF}_G(n,r) :=\big\{ \rho_G \ : \ \rho\in\Pi_{\widehat{1}}( [n]\times [r]),
 \ 
 \rho\wedge\pi=\widehat{0}\big\}
$$
the set of connected graphs whose vertex sets are non-flat partitions of
 $[n]\times [r]$, $n, r\geq 1$. 
 Next, we recall some related concepts from random graph theory, see \cite{luczakrucinski92,JLR11}.  
\begin{definition} \cite[pages~64-65]{JLR11}
  \begin{enumerate}[1)]
    \item
    A graph $G$ is balanced if 
    \begin{equation}
 \label{fjl3kjl} 
\frac{e(H)}{v(H)}\le\frac{e(G)}{v(G)},
\qquad H\subseteq G, 
  \end{equation}
 and strictly balanced if \eqref{fjl3kjl} 
 holds as a strict inequality for $H\subsetneq G$. 
\item 
 A graph $G$ with $v(G)\ge2$ is strongly balanced % , or $K_1$-balanced
 if 
\begin{equation}\label{strongbaldef}
  \frac{e(H)}{v(H)-1}\le\frac{e(G)}{v(G)-1},
  \qquad
  H\subseteq G, 
\end{equation}
and strictly strongly balanced if
\eqref{strongbaldef} holds as a strict inequality for
 $H\subsetneq G$. 
 \item 
 A graph $G$ with $v(G)\ge2$ is $K_2$-balanced
 if 
\begin{equation}\label{k2baldef}
  \frac{e(H)-1}{v(H)-2}\le\frac{e(G)-1}{v(G)-2},
  \qquad
  H\subseteq G,~ v(H)>2,
\end{equation}
and strictly $K_2$-balanced if
\eqref{k2baldef} holds as a strict inequality for
 $H\subsetneq G$. 
\end{enumerate}
\end{definition}
\begin{remark}\label{graph-example}
 As pointed out in \cite{luczakrucinski92}, strongly balanced graphs are strictly balanced. Also, trees are $K_2$-balanced but not strictly $K_2$-balanced. Cycles and complete graphs are strictly $K_2$-balanced. Moreover, every $K_2$-balanced graph except unions of disjoint edges is strongly balanced.
\end{remark}  
\begin{lemma}
  \label{restrict-partition}
  Let $n\geq 2$.
  For any connected partition 
  $\rho\in \Pi_{\widehat{1}}( [n]\times [r])$
  there exists $i\in \{1,\dots,n \}$
  such that the set partition 
  %    \label{recur-1} \rho^{(i)}:=
  $\{b\backslash\pi_i:b\in\rho\}$ 
  of $\{1,\dots,i-1,i+1,\dots,n \}\times [r]$
  is connected.
\end{lemma}
\begin{Proof}
 Let $\rho\in \Pi_{\widehat{1}}( [n]\times [r])$. 
% be a partition such that the diagram $\Gamma(\rho,\pi )$ is connected.
 We consider the connected undirected graph
 $\eufrak{g}$ on $[n]$ in which two vertices $i,j\in [n]$
 are connected if and only
 if there exists a block $b\in \rho$ such that $\pi_i \cap b \not= \emptyset$
 and $\pi_j \cap b \not= \emptyset$,
 see Figure~\ref{fig:diagram0-11}-$a)$ for an example with $n=5$.  

\begin{figure}[H]
\captionsetup[subfigure]{font=footnotesize}
\centering
\subcaptionbox{Diagram $\Gamma(\rho,\pi)$ and graph $\eufrak{g}$.}[.48\textwidth]{%
\begin{tikzpicture}
\draw[black, thick] (0,0) rectangle (5,6);

\node[anchor=east,draw, circle, inner sep=0pt, minimum size=11pt,font=\small] at (0.7,5) {1};
\node[anchor=east,draw, circle, inner sep=0pt, minimum size=11pt,font=\small] at (0.7,4) {2};
\node[anchor=east,draw, circle, inner sep=0pt, minimum size=11pt,font=\small] at (0.7,3) {3};
\node[anchor=east,draw, circle, inner sep=0pt, minimum size=11pt,font=\small] at (0.7,2) {4};
\node[anchor=east,draw, circle, inner sep=0pt, minimum size=11pt,font=\small] at (0.7,1) {5};

\node[anchor=south,font=\small] at (1,0) {1};
\node[anchor=south,font=\small] at (2,0) {2};
\node[anchor=south,font=\small] at (3,0) {3};
\node[anchor=south,font=\small] at (4,0) {4};

\filldraw [gray] (1,1) circle (2pt);
\filldraw [gray] (2,1) circle (2pt);
\filldraw [gray] (3,1) circle (2pt);
\filldraw [gray] (4,1) circle (2pt);
\filldraw [gray] (1,2) circle (2pt);
\filldraw [gray] (2,2) circle (2pt);
\filldraw [gray] (3,2) circle (2pt);
\filldraw [gray] (4,2) circle (2pt);
\filldraw [gray] (1,3) circle (2pt);
\filldraw [gray] (2,3) circle (2pt);
\filldraw [gray] (3,3) circle (2pt);
\filldraw [gray] (4,3) circle (2pt);
\filldraw [gray] (2,3) circle (2pt);
\filldraw [gray] (1,4) circle (2pt);
\filldraw [gray] (2,4) circle (2pt);
\filldraw [gray] (3,4) circle (2pt);
\filldraw [gray] (4,4) circle (2pt);
\filldraw [gray] (1,5) circle (2pt);
\filldraw [gray] (2,5) circle (2pt);
\filldraw [gray] (3,5) circle (2pt);
\filldraw [gray] (4,5) circle (2pt);

\draw[very thick] (1,5) -- (1,4); 
\draw[very thick] (3,5) -- (4,4);

\draw[very thick] (1,2) -- (1,1);
\draw[very thick] (2,2) -- (2,4);
\draw[very thick] (2,1) -- (3,2) -- (4,3) -- (3,4);

\draw[very thick,purple] (0.3,5) .. controls (0.2,4.5) .. (0.3,4);
\draw[very thick,purple] (0.3,4) .. controls (0.2,3.5) .. (0.3,3);
\draw[very thick,purple] (0.3,3) .. controls (0.2,2.5) .. (0.3,2);
\draw[very thick,purple] (0.3,2) .. controls (0.2,1.5) .. (0.3,1);
\draw[very thick,purple] (0.7,4) .. controls (0.85,3) .. (0.7,2);
\draw[very thick,purple] (0.7,4) .. controls (0.95,3) .. (0.7,1);

\end{tikzpicture}%
}
\subcaptionbox{Diagram $\Gamma(\rho,\pi)$ and spanning tree $\widebar{\eufrak{g}}$.}[.5\textwidth]{%
\begin{tikzpicture}
\draw[black, thick] (0,0) rectangle (5,6);

\node[anchor=east,draw, circle, inner sep=0pt, minimum size=11pt, font=\small] at (0.7,5) {1};
\node[anchor=east,draw, circle, inner sep=0pt, minimum size=11pt,font=\small] at (0.7,4) {2};
\node[anchor=east,draw, circle, inner sep=0pt, minimum size=11pt,font=\small] at (0.7,3) {3};
\node[anchor=east,draw, circle, inner sep=0pt, minimum size=11pt,font=\small] at (0.7,2) {4};
\node[anchor=east,draw, circle, inner sep=0pt, minimum size=11pt,font=\small] at (0.7,1) {5};

\node[anchor=south,font=\small] at (1,0) {1};
\node[anchor=south,font=\small] at (2,0) {2};
\node[anchor=south,font=\small] at (3,0) {3};
\node[anchor=south,font=\small] at (4,0) {4};

\filldraw [gray] (1,1) circle (2pt);
\filldraw [gray] (2,1) circle (2pt);
\filldraw [gray] (3,1) circle (2pt);
% \draw [very thick] (3,1) circle (4.5pt);
\filldraw [gray] (4,1) circle (2pt);
% \draw [very thick] (4,1) circle (4.5pt);
\filldraw [gray] (1,2) circle (2pt);
\filldraw [gray] (2,2) circle (2pt);
\filldraw [gray] (3,2) circle (2pt);
\filldraw [gray] (4,2) circle (2pt);
% \draw [very thick] (4,2) circle (4.5pt);
\filldraw [gray] (1,3) circle (2pt);
% \draw [very thick] (1,3) circle (4.5pt);
\filldraw [gray] (2,3) circle (2pt);
\filldraw [gray] (3,3) circle (2pt);
% \draw [very thick] (3,3) circle (4.5pt);
\filldraw [gray] (4,3) circle (2pt);
\filldraw [gray] (2,3) circle (2pt);
\filldraw [gray] (1,4) circle (2pt);
\filldraw [gray] (2,4) circle (2pt);
\filldraw [gray] (3,4) circle (2pt);
\filldraw [gray] (4,4) circle (2pt);
\filldraw [gray] (1,5) circle (2pt);
\filldraw [gray] (2,5) circle (2pt);
% \draw [very thick] (2,5) circle (4.5pt);
\filldraw [gray] (3,5) circle (2pt);
\filldraw [gray] (4,5) circle (2pt);
% \draw [very thick] (4,5) circle (4.5pt);

\draw[very thick] (1,5) -- (1,4); 
\draw[very thick] (3,5) -- (4,4);

\draw[very thick] (1,2) -- (1,1);
\draw[very thick] (2,2) -- (2,4);
\draw[very thick] (2,1) -- (3,2) -- (4,3) -- (3,4);

\draw[very thick,blue] (0.3,5) .. controls (0.2,4.5) .. (0.3,4);
% \draw[thick,purple] (0.3,4) .. controls (0.2,3.5) .. (0.3,3);
\draw[very thick,blue] (0.3,3) .. controls (0.2,2.5) .. (0.3,2);
\draw[very thick,blue] (0.3,2) .. controls (0.2,1.5) .. (0.3,1);
\draw[very thick,blue] (0.7,4) .. controls (0.85,3) .. (0.7,2);
% \draw[thick,purple] (0.7,4) .. controls (0.95,3) .. (0.7,1);

\end{tikzpicture}%
}
\caption{Example of graph $\eufrak{g}$ and its spanning tree subgraph.}
\label{fig:diagram0-11}
\end{figure}

\vspace{-.4cm}
  
\noindent
By e.g. Theorem~4.2.3 in \cite{balakrishnan},
 $\eufrak{g}$ contains a spanning
 tree $\widebar{\eufrak{g}}$, 
 as shown in Figure~\ref{fig:diagram0-11}-$b)$. 
 Let $i\in [n]$ be a leaf in the tree $\widebar{\eufrak{g}}$. 
%  (e.g. the leaf with lowest or highest index)
 If the partition 
 ${\rho}^{(i)}:=\{ b \setminus \pi_i \ \!  : \ \!  b\in \rho \}$
 of $([n] \setminus \{i\} ) \times [r]$ 
 had more than one connected component, then, for $\rho$ to be connected,
 $\pi_i$ would have to connect to all such components, 
 hence the vertex $i$ would be adjacent to
 more than one vertex in $\widebar{\eufrak{g}}$,
 which is not the case.
\end{Proof}
\begin{lemma}
  \label{lma:stronglybal}
  Let $G$ be a graph with $v(G)=r$ vertices, $r \geq 2$.
  If $G$ is strongly balanced and connected, then for any $n\ge1$
  we have
  \begin{equation}
    \label{stronglybalin}
      \frac{e(G)}{v(G)-1}\le\frac{e(\rho_G)}{v(\rho_G)-1},
      \qquad
  \rho_G\in {\rm CNF}_G(n,r).
    \end{equation}
 \end{lemma}
\begin{Proof}
% We let $r=v(G)\ge2$ and $k=e(G)$.
  The claim holds for $n=1$, as in this case ${\rm CNF}_G(1,r)$
  contains only one graph, which is isomorphic to $G$.
  Suppose now that \eqref{stronglybalin} holds up to the rank $n\geq 1$.
  Let $\rho$ be a non-flat connected partition on $[n+1]\times[r]$
  with associated graph $\rho_G\in{\rm CNF}_G(n,r)$.
  By Lemma~\ref{restrict-partition}, up to reordering
  of $\{1,\ldots , n+1\}$ there exists 
  $\widebar{\rho}_G\in {\rm CNF}_G(n,r)$
  induced by $\rho_G$ on
     $$
     V(\widebar{\rho}_G):=
     \big\{
     b\in \rho \ : \ b \cap (\pi_1\cup \cdots \cup \pi_n) \not= \emptyset
     \big\}. 
     $$
 Let 
     $\widehat{\rho}_G\in
     {\rm CNF}_G(1,r)$
     denote the subgraph 
  induced by $\rho_G$ on
     $$
     V(\widehat{\rho}_G):=
     \big\{
     b\in \rho \ : \ b \cap \pi_{n+1} \not= \emptyset
     \big\}, 
     $$
     with
     $\widehat{\rho}_G \simeq G$ because $\rho$ is non-flat, 
     and let $H:=\widebar{\rho}_G\cap \widehat{\rho}_G$.
     Since $H\subseteq \widehat{\rho}_G$,
     by \eqref{strongbaldef} we have 
       \begin{align*}
  \frac{e(H)}{v(H)-1}\le\frac{e(G)}{v(G)-1}. 
  \end{align*} 
 Hence, by the induction
 hypothesis \eqref{stronglybalin} applied at the rank $n$
 to   $\widebar{\rho}_G\in {\rm CNF}_G(n,r)$, we have 
         \begin{align*}
         \frac{e(\rho_G)}{v(\rho_G)-1}
         & =
         \frac{e(\widebar{\rho}_G)+e(\widehat{\rho}_G)-e(H)}{
           v(\widebar{\rho}_G)+v(\widehat{\rho}_G)-v(H)-1}
         \\
         & =
         \frac{e(\widebar{\rho}_G)+e(G)-e(H)}{
           v(\widebar{\rho}_G)+v(G)-v(H)-1}
         \\
                  & \geq 
         \frac{
          e(G)(v(\widebar{\rho}_G)-1) / ( v(G)-1 )  
          +e(G)-
          e(G)(v(H)-1) / (v(G)-1)}{
           v(\widebar{\rho}_G)+v(G)-v(H)-1}
         \\
                  & = 
         \frac{e(G)}{v(G)-1}. 
         \end{align*} 
\end{Proof}
% We also note that if $G$ is connected and the inequality \eqref{stronglybalin} holds for all $\rho_G\in{\rm CNF}_G(2,r)$, then $G$ is strongly balanced.
\subsection*{Random graph regimes} 
\noindent
First, we recall the following standard notation for the asymptotic behavior of the relative order of magnitude of two functions $f(\lambda )$ and $g(\lambda )$ as $\lambda$ tends to infinity.
We write $f(\lambda )=O(g(\lambda ))$ if $\limsup_{\lambda \to\infty} f(\lambda ) / g(\lambda ) <\infty$;  $f(\lambda )=\Omega(g(\lambda ))$ if $\liminf_{\lambda \to\infty} f(\lambda ) / g(\lambda )>0$, and $f(\lambda )=\Theta(g(\lambda ))$ if $f(\lambda )=O(g(\lambda ))$ and $f(\lambda )=\Omega(g(\lambda ))$. We also write
\begin{itemize}
    \item $f(\lambda )\asymp g(\lambda )$ if $f(\lambda )=\Theta(g(\lambda ))$;
    \item $f(\lambda )\sim g(\lambda )$ if $\lim_{\lambda \to \infty} f(\lambda )/g(\lambda ) = 1$, 
%    \item $f(x)=o(g(x))$, if $f(x)/g(x)\to 0$;
%    \item $f(x)\ll g(x)$, or $g(x)\gg f(x)$ if $f(x)\ge0$ and $f(x)=o(g(x))$.
\end{itemize}
 and use the convention $0/0=0.$
%According to \cite{LiuPrivault}, the regime is called the dilute regime if $c_\lambda\asymp\lambda^{-\alpha}$ for some $\alpha\in[0,1)$, and the sparse regime if $c_\lambda\asymp\lambda^{-\alpha}$ for some $\alpha>1$.
\begin{definition}
  Let $G$ be a graph with $v(G)=r$ vertices, $r \geq 2$.
  We consider the dilute, full random graphs
  and sparse regimes, defined
  as follows.  
\begin{itemize}
\item Dilute regime:
$\lambda c_\lambda^{e(G)/ ( r-1 )}\gg1$. 
\item Full random graphs regime:
$\lambda c_\lambda^{e(G)/ ( r-1 )}\asymp 1$. 
  \item Sparse regime: $\lambda c_\lambda^{e(G)/ ( r-1 )}\ll1$. 
\end{itemize}
\end{definition}
 We note that when 
 $c_\lambda\sim\lambda^{-\varepsilon}$ for some
 $\varepsilon < (r-1)/e(G)\leq 1$ 
 we are in the dilute regime,
 whereas when $c_\lambda\sim \lambda^{-\varepsilon}$
 for some $\varepsilon > (r-1)/e(G)$
 we are in the sparse regime.
% When $c_\lambda\equiv K$ for all $\lambda>0$, the dilute regime is also called the full random graph regime. 
\section{Main result} 
\noindent 
For a given graph $G$, finding the dominant asymptotic order
in the expression of $\kappa_n(N_G)$, $n\ge1$,
is key to the derivation of normal approximation results
via the cumulant method. 
 We have 
\begin{equation}
  \label{cum-eq}
  \kappa_n(N_G)=
  \sum_{\rho \in \Pi_{\widehat{1}} ( [n] \times [r])
    \atop
    {\rho \wedge \pi = \widehat{0}
    \atop {\rm (non-flat \ \! connected)}
    }
    }
F(\rho)
\end{equation} 
 with 
\begin{equation}
\nonumber
  F(\rho):= \int_{(\R^d )^{|\rho |}}
  \Bigg(
\prod_{(\eta_1,\eta_2) \in E(\rho_G)}
H_\lambda(x_{\eta_1},x_{\eta_2})
\Bigg)
\prod_{\eta \in V(\rho_G)}
\Lambda_{\lambda} (\mathrm{d}x_\eta ),
\qquad \rho\in\Pi( [n]\times [r]), 
\end{equation}
 and 
 \begin{equation}
   \label{asymptics-1}
F(\rho)=\Theta(\lambda^{|\rho|}c_\lambda^{e(\rho_G)}). 
\end{equation}
In what follows, we identify the leading contribution
appearing in \eqref{cum-eq} by a geometric argument on 
$\inte \times \inte$.
\begin{definition}\label{xyplot}
 For $n\ge2$ we let 
\begin{equation}
\nonumber
    \Sigma_n(G):=\{(nv(G)-v(\rho_G),ne(G)-e(\rho_G)) \ : \ \rho_G\in
        {\rm CNF}_G(n,r)
        \} \subseteq \inte\times \inte.
\end{equation}
% We call the upper boundary $\widehat{\Sigma}_n(G)$ of the convex hull of $\Sigma_n(G)$ the {\it roof}, and 
\end{definition}
We note that the upper boundary $\widehat{\Sigma}_n(G)$ of the convex hull of $\Sigma_n(G)$ starts at $(n-1,0)$ and ends at $((n-1)r,(n-1)e(G))$,
 and we say that a diagram $\rho_G\in {\rm CNF}_G(n,r)$ lies on the boundary $\widehat{\Sigma}_n(G)$ if the point $(nv(G)-v(\rho_G),ne(G)-e(\rho_G))$ does. 
The next figures can be plotted after loading the code presented in
appendix and running the following commands.

\begin{table}[H] 
  \centering
\scriptsize %   \small
 \resizebox{0.9\textwidth}{!}
    {
  \begin{tabular}{|ll|l|} % {\textwidth}{|XX|XX|}
 \hline
 \multicolumn{2}{|l|}{
\EscVerb{G = [[1,2],[2,3],[3,1]]; L=convexhull(2,G)}
 }
 \\
 \hline
 \multicolumn{2}{|l|}{
\EscVerb{Polyhedron(L).plot(color = "pink")+point(L,color = "blue",size=20)}
 }
 \\
\hline
\end{tabular}
}
% \caption{Cumulants of the count of $2$-hop paths with two endpoints in dimension $d=1$.}
\label{t1}
\end{table} 

\vspace{-0.4cm}

\noindent 
For example, we have  $\Sigma_2(G)=\{(3,3),(2,1),(1,0)\}$
 when $G = C_3$ is a triangle. 
Figure~\ref{fig:diagram3} presents an illustration for
$\Sigma_n(C_3)$ with $n=3,4$. 

\begin{figure}[H]
  \begin{subfigure}{.5\textwidth}
    \vskip1cm
\begin{tikzpicture}
%\draw[black, thick] (0,0) rectangle (7,4);
\draw[->] (0,0) -- (0,4);
\node[anchor=east,font=\tiny] at (-0.05,4) {y};
\draw[->] (0,0) -- (7,0);
\node[anchor=north,font=\tiny] at (7,-0.05) {x};
\draw[thick] (-0.05,0.5) -- (0.05,0.5);
\node[anchor=east,font=\tiny] at (-0.05,0.5) {1};
\draw[thick] (-0.05,1) -- (0.05,1);
\node[anchor=east,font=\tiny] at (-0.05,1) {2};
\draw[thick] (-0.05,2) -- (0.05,2);
\node[anchor=east,font=\tiny] at (-0.05,2) {4};
\draw[thick] (-0.05,3) -- (0.05,3);
\node[anchor=east,font=\tiny] at (-0.05,3) {6};
\draw[thick] (-0.05,1.5) -- (0.05,1.5);
\node[anchor=east,font=\tiny] at (-0.05,1.5) {3};
\draw[thick] (-0.05,2.5) -- (0.05,2.5);
\node[anchor=east,font=\tiny] at (-0.05,2.5) {5};
\node[anchor=east,font=\tiny] at (-0.05,3) {6};
\draw[thick] (1,-0.05) -- (1,0.05);
\node[anchor=north,font=\tiny] at (1,-0.05) {1};
\draw[thick] (2,-0.05) -- (2,0.05);
\node[anchor=north,font=\tiny] at (2,-0.05) {2};
\draw[thick] (3,-0.05) -- (3,0.05);
\node[anchor=north,font=\tiny] at (3,-0.05) {3};
\draw[thick] (4,-0.05) -- (4,0.05);
\node[anchor=north,font=\tiny] at (4,-0.05) {4};
\draw[thick] (5,-0.05) -- (5,0.05);
\node[anchor=north,font=\tiny] at (5,-0.05) {5};
\draw[thick] (6,-0.05) -- (6,0.05);
\node[anchor=north,font=\tiny] at (6,-0.05) {6};
\filldraw [blue] (2,0) circle (1.5pt);
\filldraw [blue] (3,0) circle (1.5pt);
\filldraw [blue] (3,0.5) circle (1.5pt);
\filldraw [blue] (4,0.5) circle (1.5pt);
\filldraw [blue] (4,1) circle (1.5pt);
\filldraw [blue] (5,1.5) circle (1.5pt);
\filldraw [blue] (4,1.5) circle (1.5pt);
\filldraw [blue] (5,2) circle (1.5pt);
\filldraw [blue] (6,3) circle (1.5pt);
\draw[thick,red] (2,0) -- (6,3);
    \end{tikzpicture}
    \caption{Plot of $\Sigma_3(C_3)$ with $\widehat{\Sigma}_3(C_3)$ in red.}
    \label{fig:diagram3}
\end{subfigure}
\begin{subfigure}{.5\textwidth}
\begin{tikzpicture}
\draw[->] (0,0) -- (0,5);
\node[anchor=east,font=\tiny] at (-0.05,5) {y};
\draw[->] (0,0) -- (5,0);
\node[anchor=north,font=\tiny] at (5,-0.05) {x};
\draw[thick] (-0.05,0.5) -- (0.05,0.5);
\node[anchor=east,font=\tiny] at (-0.05,0.5) {1};
\draw[thick] (-0.05,1) -- (0.05,1);
\node[anchor=east,font=\tiny] at (-0.05,1) {2};
\draw[thick] (-0.05,1.5) -- (0.05,1.5);
\node[anchor=east,font=\tiny] at (-0.05,1.5) {3};
\draw[thick] (-0.05,2) -- (0.05,2);
\node[anchor=east,font=\tiny] at (-0.05,2) {4};
\draw[thick] (-0.05,2.5) -- (0.05,2.5);
\node[anchor=east,font=\tiny] at (-0.05,2.5) {5};
\draw[thick] (-0.05,3) -- (0.05,3);
\node[anchor=east,font=\tiny] at (-0.05,3) {6};
\draw[thick] (-0.05,3.5) -- (0.05,3.5);
\node[anchor=east,font=\tiny] at (-0.05,3.5) {7};
\draw[thick] (-0.05,4) -- (0.05,4);
\node[anchor=east,font=\tiny] at (-0.05,4) {8};
\draw[thick] (-0.05,4.5) -- (0.05,4.5);
\node[anchor=east,font=\tiny] at (-0.05,4.5) {9};
\draw[thick] (1,-0.05) -- (1,0.05);
\node[anchor=north,font=\tiny] at (1,-0.05) {2};
\draw[thick] (1.5,-0.05) -- (1.5,0.05);
\node[anchor=north,font=\tiny] at (1.5,-0.05) {3};
\draw[thick] (2,-0.05) -- (2,0.05);
\node[anchor=north,font=\tiny] at (2,-0.05) {4};
\draw[thick] (2.5,-0.05) -- (2.5,0.05);
\node[anchor=north,font=\tiny] at (2.5,-0.05) {5};
\draw[thick] (3,-0.05) -- (3,0.05);
\node[anchor=north,font=\tiny] at (3,-0.05) {6};
\draw[thick] (3.5,-0.05) -- (3.5,0.05);
\node[anchor=north,font=\tiny] at (3.5,-0.05) {7};
\draw[thick] (4,-0.05) -- (4,0.05);
\node[anchor=north,font=\tiny] at (4,-0.05) {8};
\draw[thick] (4.5,-0.05) -- (4.5,0.05);
\node[anchor=north,font=\tiny] at (4.5,-0.05) {9};
\filldraw [blue] (1.5,0) circle (1.5pt);
\filldraw [blue] (2,0) circle (1.5pt);
\filldraw [blue] (2.5,0) circle (1.5pt);
\filldraw [blue] (3,0) circle (1.5pt);
\filldraw [blue] (2,0.5) circle (1.5pt);
\filldraw [blue] (2.5,0.5) circle (1.5pt);
\filldraw [blue] (3,0.5) circle (1.5pt);
\filldraw [blue] (3.5,1) circle (1.5pt);
\filldraw [blue] (2.5,1) circle (1.5pt);
\filldraw [blue] (3.5,1.5) circle (1.5pt);
\filldraw [blue] (2.5,1.5) circle (1.5pt);
\filldraw [blue] (3,1.5) circle (1.5pt);
\filldraw [blue] (3,2) circle (1.5pt);
\filldraw [blue] (3.5,2) circle (1.5pt);
\filldraw [blue] (3.5,2.5) circle (1.5pt);
\filldraw [blue] (4,3) circle (1.5pt);
\filldraw [blue] (3.5,3) circle (1.5pt);
\filldraw [blue] (4,3.5) circle (1.5pt);
\filldraw [blue] (4.5,4.5) circle (1.5pt);
\draw[thick,red] (1.5,0) -- (4.5,4.5);
\end{tikzpicture}%
\caption{Plot of $\Sigma_4(C_3)$ with $\widehat{\Sigma}_4(C_3)$ in red.}
\label{fig:diagram3-b}
\end{subfigure}
\caption{Set $\Sigma_n(C_3)$ and upper boundary of its convex hull for $n=3,4$.}
\label{fig1}
\end{figure}
\noindent
In Figure~\ref{fig:diagram3} we have 
$$\Sigma_3(C_3)=\{(6, 6), (5, 4), (4, 3), (5, 3), (4, 2), (4, 1), (3, 1), (3, 0), (2, 0)\},$$
 while in Figure~\ref{fig:diagram3-b} we have 
\begin{align*}
\Sigma_4(C_3)&=\{(9, 9), (8, 7), (7, 6), (8, 6), (7, 5), (7, 4), (6, 4), (6, 3),(5, 3), (7, 3), \\
&~~~~ (6, 2),
(5, 2), (7, 2), (6, 1), (5, 1), (4, 1), (6, 0), (5, 0), (4, 0), (3, 0)\}.
\end{align*}
% \vspace{-0.4cm}
\noindent
The case where $G=C_4$ is a rectangle is considered
in Figure~\ref{fig:diagram4} with $\Sigma_2(C_4)$ and in 
Figure~\ref{fig:diagram4-b} with $\Sigma_3(C_4)$. 

% \smallskip
 
\begin{figure}[H]
% \captionsetup[subfigure]{font=footnotesize}
% \centering
\begin{subfigure}{.5\textwidth}
  \vskip2cm
  \begin{tikzpicture}
%\draw[black, thick] (0,0) rectangle (7,4);
\draw[->] (0,0) -- (0,3);
\node[anchor=east,font=\tiny] at (-0.05,3) {y};
\draw[->] (0,0) -- (5,0);
\node[anchor=north,font=\tiny] at (5,-0.05) {x};
\draw[thick] (-0.05,0.5) -- (0.05,0.5);
\node[anchor=east,font=\tiny] at (-0.05,0.5) {1};
\draw[thick] (-0.05,1) -- (0.05,1);
\node[anchor=east,font=\tiny] at (-0.05,1) {2};
\draw[thick] (-0.05,2) -- (0.05,2);
\node[anchor=east,font=\tiny] at (-0.05,2) {4};
\draw[thick] (1,-0.05) -- (1,0.05);
\node[anchor=north,font=\tiny] at (1,-0.05) {1};
\draw[thick] (2,-0.05) -- (2,0.05);
\node[anchor=north,font=\tiny] at (2,-0.05) {2};
\draw[thick] (3,-0.05) -- (3,0.05);
\node[anchor=north,font=\tiny] at (3,-0.05) {3};
\draw[thick] (4,-0.05) -- (4,0.05);
\node[anchor=north,font=\tiny] at (4,-0.05) {4};
\filldraw [blue] (1,0) circle (1.5pt);
\filldraw [blue] (2,0) circle (1.5pt);
\filldraw [blue] (2,0.5) circle (1.5pt);
\filldraw [blue] (3,0.5) circle (1.5pt);
\filldraw [blue] (3,1) circle (1.5pt);
\filldraw [blue] (4,1) circle (1.5pt);
\filldraw [blue] (4,2) circle (1.5pt);
\draw[thick,red] (1,0) -- (4,2);
\end{tikzpicture}%
    \caption{Plot of $\Sigma_2(C_4)$ with $\widehat{\Sigma}_2(C_4)$ in red.}
    \label{fig:diagram4}
\end{subfigure} 
\begin{subfigure}{.5\textwidth}
\begin{tikzpicture}
\draw[->] (0,0) -- (0,5);
\node[anchor=east,font=\tiny] at (-0.05,5) {y};
\draw[->] (0,0) -- (5,0);
\node[anchor=north,font=\tiny] at (5,-0.05) {x};
\draw[thick] (-0.05,0.5) -- (0.05,0.5);
\node[anchor=east,font=\tiny] at (-0.05,0.5) {1};
\draw[thick] (-0.05,1) -- (0.05,1);
\node[anchor=east,font=\tiny] at (-0.05,1) {2};
\draw[thick] (-0.05,1.5) -- (0.05,1.5);
\node[anchor=east,font=\tiny] at (-0.05,1.5) {3};
\draw[thick] (-0.05,2) -- (0.05,2);
\node[anchor=east,font=\tiny] at (-0.05,2) {4};
\draw[thick] (-0.05,2.5) -- (0.05,2.5);
\node[anchor=east,font=\tiny] at (-0.05,2.5) {5};
\draw[thick] (-0.05,3) -- (0.05,3);
\node[anchor=east,font=\tiny] at (-0.05,3) {6};
\draw[thick] (-0.05,4) -- (0.05,4);
\node[anchor=east,font=\tiny] at (-0.05,4) {8};
\draw[thick] (1,-0.05) -- (1,0.05);
\node[anchor=north,font=\tiny] at (1,-0.05) {2};
\draw[thick] (2,-0.05) -- (2,0.05);
\node[anchor=north,font=\tiny] at (2,-0.05) {4};
\draw[thick] (3,-0.05) -- (3,0.05);
\node[anchor=north,font=\tiny] at (3,-0.05) {6};
\draw[thick] (4,-0.05) -- (4,0.05);
\node[anchor=north,font=\tiny] at (4,-0.05) {8};
\filldraw [blue] (1,0) circle (1.5pt);
\filldraw [blue] (1.5,0) circle (1.5pt);
\filldraw [blue] (2,0) circle (1.5pt);
\filldraw [blue] (2.5,0.5) circle (1.5pt);
\filldraw [blue] (2.5,1) circle (1.5pt);
\filldraw [blue] (2.5,1.5) circle (1.5pt);
\filldraw [blue] (2.5,2) circle (1.5pt);
\filldraw [blue] (2,0.5) circle (1.5pt);
\filldraw [blue] (2,1) circle (1.5pt);
\filldraw [blue] (1.5,0.5) circle (1.5pt);
\filldraw [blue] (3,2.5) circle (1.5pt);
\filldraw [blue] (3.5,3) circle (1.5pt);
\filldraw [blue] (3.5,2.5) circle (1.5pt);
\filldraw [blue] (3.5,1.5) circle (1.5pt);
\filldraw [blue] (3.5,1) circle (1.5pt);
\filldraw [blue] (3,1) circle (1.5pt);
\filldraw [blue] (3,.5) circle (1.5pt);
\filldraw [blue] (3,0) circle (1.5pt);
\filldraw [blue] (2.5,0) circle (1.5pt);
\filldraw [blue] (3,1.5) circle (1.5pt);
\filldraw [blue] (3,2) circle (1.5pt);
\filldraw [blue] (3.5,2) circle (1.5pt);
\filldraw [blue] (4,3) circle (1.5pt);
\filldraw [blue] (4,4) circle (1.5pt);
\draw[thick,red] (1,0) -- (4,4);
\end{tikzpicture}%
\caption{Plot of $\Sigma_3(C_4)$ with $\widehat{\Sigma}_3(C_4)$ in red.}
% \caption{  Example of $\Sigma_n(C_4)$
  % on $G=(V_G,E)$ with $V_G=\{u_1, u_2,u_3,u_4,v_1,v_2\}$ 
% with $n=2,3$.}
\label{fig:diagram4-b}
\end{subfigure}
\caption{Set $\Sigma_n(C_4)$ and upper boundary of its convex hull for $n=2,3$.}\label{fig2}
\end{figure}
\vskip-0.3cm
\noindent 
In the sequel, we let $\widehat{\Sigma}_n(G)$ be written as 
 $$
 \widehat{\Sigma}_n(G)=\{(x_i,y_i):i=0,1,\dots,\ell\}, 
$$ 
 for some $\ell\ge1$,
with $(x_0,y_0)=(n-1,0)$,
$(x_\ell,y_\ell)=((n-1)r,(n-1) e(G))$,
and
$$x_0<x_1 < \cdots < x_\ell < x_{\ell +1}:= + \infty.
$$
In addition, for any point $(x_i,y_i)$ on the boundary
$\widehat{\Sigma}_n(G)$, $i\in \{0,1,\ldots , \ell\}$,
 there exists $\rho_{i,G}\in {\rm CNF}_G(n,r)$
 such that
 \begin{equation}
   \label{fjklf} 
v(\rho_{i,G})=nr-x_i \quad \mbox{and} \quad e(\rho_{i,G})=ne(G)-y_i. 
\end{equation}

\begin{definition}
 For any $\rho_G \in \Sigma_n(G)$, we let 
$$
 x_{\rho_G} : = nv(G)-v(\rho_G) \mbox{ and }
 \ 
 y_{\rho_G} : = ne(G)-e(\rho_G), 
$$ 
 and given $i\in \{0,\ldots , \ell \}$ such that
 $x_i\le x_{\rho_G} <x_{i+1}$, we let 
 $$
 \theta^-_{\rho_G} : = \left\{
 \begin{array}{ll}
   +\infty & \mbox{if } i=0,
   \medskip
   \\ 
   \displaystyle
   \frac{y_i-y_{i-1}}{x_i-x_{i-1}} & \mbox{if } i\in \{1,\ldots , \ell \}, 
 \end{array}
 \right.
 \qquad 
 \theta^+_{\rho_G} : = 
 \left\{
 \begin{array}{ll}
   \displaystyle
   \frac{y_{i+1}-y_i}{x_{i+1}-x_i}
   & \mbox{if } i\in \{0,\ldots ,  \ell - 1\}, 
   \medskip
   \\ 
   0 & \mbox{if } i=\ell. 
 \end{array}
 \right.
$$ 
\end{definition} 
Since $\widehat{\Sigma}_n(G)$ is the
upper boundary of
the convex hull of $\Sigma_n(G)$, we note that
$\theta^-_{\rho_G} \geq \theta^+_{\rho_G}$.
 For a given $n\ge2$, a diagram $\rho_G\in {\rm CNF}_G(n,r)$
 is called a leading diagram if 
\begin{equation}\label{asy-2}
\kappa_n(N_G)=\Theta\big(\lambda^{v(\rho_G)}c_\lambda^{e(\rho_G)}\big). 
\end{equation}
% if $\lim_{\lambda \to \infty} c_\lambda =0$. 
% Leading diagrams owes their name to the fact that they correspond to the leading terms of the cumulant expression \eqref{cum-eq}.
% The terminology is extending from random graph theory, \cite{luczakrucinski92,JLR}, where it is defined only for variance term, i.e. when $n=2$.
 In the next lemma, we identify leading diagrams
 by a convexity argument. 
% by plotting the points that represent a `loss' caused by merging vertices and gluing edges when generating diagram $\rho_G$.
\begin{lemma}\label{lm:slope}
  Assume that $\lim_{\lambda \to \infty} c_\lambda =0$.
  Let $G$ be a connected graph with $v(G)=r$ vertices, $r\ge2$, and
  let $n\ge2$. 
  \begin{enumerate}[1)]
  \item
    Any leading diagram $\rho_G\in {\rm CNF}_G(n,r)$ lies on the
    upper boundary $\widehat{\Sigma}_n(G)$, i.e. 
  $$
  (nv(G)-v(\rho_G),ne(G)-e(\rho_G)) \in \widehat{\Sigma}_n(G).
  $$
 \item
  If a diagram $\rho_G\in {\rm CNF}_G(n,r)$ lies on the
  upper boundary $\widehat{\Sigma}_n(G)$ and 
  \begin{equation}
    \label{fjk5lf4} 
  \lambda c_\lambda^{ \theta^-_{\rho_G}}=O(1) \ \mbox{ and } \
  \lambda c_\lambda^{ \theta^+_{\rho_G} }=\Omega(1), 
\end{equation}
 then $\rho_G$ is a leading diagram.
\end{enumerate}
\end{lemma}
\begin{Proof}
  % It is clear that $(n-1,0)$ and $((n-1)r,(n-1)k)$ lies on the boundary $\widehat{\Sigma}_n(G)$.
\noindent
$1)$ Consider a diagram
$\rho_G\in {\rm CNF}_G(n,r)$ which is not on the boundary
$\widehat{\Sigma}_n(G)$, i.e. 
$$
( x_{\rho_G} , y_{\rho_G} ) :=
( nv(G)-v(\rho_G) , ne(G)-e(\rho_G) )
\in\Sigma_n(G)\backslash \widehat{\Sigma}_n(G).
$$ 
 If $x_{\rho_G}=x_\ell$, we know that $y_{\rho_G}<y_\ell,$ since $(x_{\rho_G},y_{\rho_G})$ is not on the boundary $\widehat{\Sigma}_n(G)$. Therefore, we have 
 $$\lambda^{v(\rho_G)-v(\rho_{\ell,G})}c_\lambda^{e(\rho_G)-e (\rho_{\ell,G})}=c_\lambda^{y_{\ell}-y_{\rho_G}}\ll1,
 $$
 and $\rho_G$ is not a leading diagram.

 \medskip

 If $x_{\rho_G}<x_\ell$, we choose $i\in \{0,\ldots , \ell-1 \}$
such that
 $x_i\le x_{\rho_G} <x_{i+1}$. 
Since $\widehat{\Sigma}_n(G)$ is the upper boundary of a
convex hull, by \eqref{fjklf} 
 if $i<\ell $ we have 
 $$\nu:=\frac{y_{i+1}-y_{\rho_G}}{x_{i+1}-x_{\rho_G}}
 > \theta^+_{\rho_G} = \frac{y_{i+1}-y_i}{x_{i+1}-x_i}
 = \frac{e(\rho_{i,G})-e(\rho_{i+1,G})}{v(\rho_{i,G})-v(\rho_{i+1,G})},
 $$
 i.e. $\lambda c_\lambda^\nu
 \ll\lambda c_\lambda^{\theta^+_{\rho_G}}$
 because $\lim_{\lambda \to \infty} c_\lambda =0$, and 
 we consider three cases.
 \begin{enumerate}[i)]
   \item If
 $
1\ll\lambda c_\lambda^\nu
$, we have 
\begin{align*}
  \lambda^{v({\rho_G})-v(\rho_{i+1,G})}c_\lambda^{e({\rho_G})-e(\rho_{i+1,G})}
  & =  \left(\lambda c_\lambda^\nu\right)^{v({\rho_G})-v(\rho_{i+1,G})}
  \\
  & \ll \left(\lambda c_\lambda^{\theta^+_{\rho_G}} \right)^{v({\rho_G})-v(\rho_{i+1,G})}
  \\
   &   \leq \left(\lambda c_\lambda^{\theta^+_{\rho_G}}\right)^{v(\rho_{i,G})-v(\rho_{i+1,G})}
  \\
   & =  \lambda^{v(\rho_{i,G})-v(\rho_{i+1,G})}c_\lambda^{e(\rho_{i,G})-e(\rho_{i+1,G})}, 
\end{align*}
which implies $\lambda^{v({\rho_G})}c_\lambda^{e({\rho_G})}\ll \lambda^{v(\rho_{i,G})}c_\lambda^{e(\rho_{i,G})}$
 as $\lim_{\lambda \to \infty} c_\lambda =0$, hence
 the diagram $\rho_G$ is not leading. 
\item  If
 $\lim_{\lambda \to \infty} \lambda c_\lambda^\nu = c>0$, we have 
\begin{align*}
  \lambda^{v({\rho_G})-v(\rho_{i+1,G})}c_\lambda^{e({\rho_G})-e(\rho_{i+1,G})}
  & =  \left(\lambda c_\lambda^\nu\right)^{v({\rho_G})-v(\rho_{i+1,G})}
  \\
  & \ll \left(\lambda c_\lambda^{\theta^+_{\rho_G}} \right)^{v({\rho_G})-v(\rho_{i+1,G})}
  \\
   &  \asymp \left(\lambda c_\lambda^{\theta^+_{\rho_G}}\right)^{v(\rho_{i,G})-v(\rho_{i+1,G})}
  \\
   & =  \lambda^{v(\rho_{i,G})-v(\rho_{i+1,G})}c_\lambda^{e(\rho_{i,G})-e(\rho_{i+1,G})}, 
\end{align*}
\noindent
 and we conclude as above. 
\item  If
 $\lambda c_\lambda^\nu\ll 1$, we have
 \begin{equation*}
\lambda^{v({\rho_G})-v(\rho_{i+1,G})}c_\lambda^{e({\rho_G})-e(\rho_{i+1,G})}\ll1, 
\end{equation*}
 hence
 $$\lambda^{v({\rho_G})}c_\lambda^{e({\rho_G})}\ll \lambda^{v(\rho_{i+1,G})}c_\lambda^{e(\rho_{i+1,G})}.
 $$
\noindent 
 As a consequence of the above, we find  
\begin{equation}
\nonumber
\lambda^{v({\rho_G})}c_\lambda^{e({\rho_G})}
\ll
\lambda^{v(\rho_{i,G})}c_\lambda^{e(\rho_{i,G})}
  \ \mbox{ or } \ 
\lambda^{v({\rho_G})}c_\lambda^{e({\rho_G})}
\ll
\lambda^{v(\rho_{i+1,G})}c_\lambda^{e(\rho_{i+1,G})}, 
\end{equation}
 hence ${\rho_G}$ is not a leading diagram. 
\end{enumerate} 
\noindent
$2)$ Suppose that $\rho_{G}$ lies on the boundary $\widehat{\Sigma}_{n}(G)$.
Then, there exists $i\in \{0,\dots,\ell\}$
 such that $(x_{\rho_{G}},y_{\rho_{G}})=(x_{i},y_{i})$, and it holds that
$$
\lambda c_\lambda^{ \theta^-_{\rho_G}}=O(1) \ \mbox{ and } \
  \lambda c_\lambda^{ \theta^+_{\rho_G} }=\Omega(1). 
$$
  \begin{enumerate}[i)] 
  \item
    If $j<i$, then $x_{j}-x_{i}<0$ and
  $$
  \frac{y_{i}-y_{j}}{x_{i}-x_{j}}\ge\frac{y_{i}-y_{i-1}}{x_{i}-x_{i-1}}=\theta_{\rho_{G}}^{-}.
  $$
  Hence, 
\begin{equation}
  \lambda c_{\lambda}^{(y_{i}-y_{j})/(x_{i}-x_{j})}=O\big(\lambda c_\lambda^{ \theta^-_{\rho_G}}
  \big)=O(1).
\end{equation}
Now, since
\begin{eqnarray}
  \label{fjkl4}
  \frac{\lambda^{nv(G)-x_{i}}c_{\lambda}^{ne(G)-y_{i}}}{\lambda^{nv(G)-x_{j}}c_{\lambda}^{ne(G)-y_{j}}}=\lambda^{x_{j}-x_{i}}c_{\lambda}^{y_{j}-y_{i}}=\big(\lambda c_{\lambda}^{
    (y_{j}-y_{i})/(x_{j}-x_{i})}\big)^{x_{j}-x_{i}}, 
\end{eqnarray} 
 we find 
\begin{equation}
  \nonumber
  \frac{\lambda^{v(\rho_G)}c_{\lambda}^{e(\rho_G)}}{\lambda^{v(\rho_{j,G})}c_{\lambda}^{e(\rho_{j,G})}}=\Omega(1).
\end{equation}
\item
   If $j>i$, then $x_{j}-x_{i}>0$ and
$$
\frac{y_{j}-y_{i}}{x_{j}-x_{i}}\le\theta_{\rho_{G}}^{+},
$$
 therefore
\begin{equation}
  \lambda c_{\lambda}^{ ( y_{i}-y_{j})/(x_{i}-x_{j})}=\Omega
 \big(\lambda c_\lambda^{ \theta^+_{\rho_G}}\big)=\Omega(1),
\end{equation}
and \eqref{fjkl4} shows that 
\begin{equation}
  \nonumber
  \frac{\lambda^{v(\rho_G)}c_{\lambda}^{e(\rho_G)}}{\lambda^{v(\rho_{j,G})}c_{\lambda}^{e(\rho_{j,G})}}=\Omega(1).
\end{equation}
  \end{enumerate}
  This ensures that $\rho_G$ is a leading diagram.
\end{Proof}
 By Lemma~\ref{lma:stronglybal} we obtain the following dichotomy result. 
\begin{prop}
   \label{dich1}
 Let $G$ be a strongly balanced and connected
 graph with $v(G)=r$ vertices, $r \geq 2$,
 and assume that $\lim_{\lambda \to \infty} c_\lambda =0$.
\begin{enumerate}[a)]
  \item If $\lambda c_\lambda^{e(G)/ ( r-1 )}\gg1$, we have
\begin{equation}
\label{r1} 
  \kappa_n(N_G)=\Theta\big(\lambda^{(r-1)n+1}c_\lambda^{ne(G)}\big). 
\end{equation}
\item If $\lambda c_\lambda^{e(G)/ ( r-1 )}\ll1$, we find 
\begin{equation}
\label{r2} 
  \kappa_n(N_G)=\Theta\big(\lambda^{r}c_\lambda^{e(G)}\big).
\end{equation}
\item If $\lambda c_\lambda^{e(G)/ ( r-1 )}\asymp 1$, we have
\begin{equation}
\label{r3} 
  \kappa_n(N_G)\asymp \lambda.  
\end{equation}
\end{enumerate} 
\end{prop} 
\begin{Proof}
  It is clear that $v(\rho_G)\le (r-1)n+1$ for all $\rho_G\in {\rm CNF}_G(n,r)$. For any $\rho_G$ with $v(\rho_G)<(r-1)n+1$,
 Lemma~\ref{lma:stronglybal} shows that 
\begin{equation}\label{condi-1}
  \frac {e(G)}{v(G)-1}\le\frac{e(\rho_G)}{v(\rho_G)-1},
  \qquad
  \rho_G\in {\rm CNF}_G(n,r), 
\end{equation}
 which implies 
\begin{equation}
\nonumber
          \frac{ne(G)-e(\rho_G)}{(r-1)n+1-v(\rho_G)}\le\frac{e(G)}{r-1}, 
\end{equation}
 hence 
$\widehat{\Sigma}_n(G)$ 
 is the segment with endpoints
 $$
 (x_0,y_0)=(n-1,0)
 \ \mbox{ and } \ 
 (x_\ell,y_\ell)=((n-1)r,(n-1) e(G)),
 $$ 
 and slope
 $
 \theta^+_{\rho_G}
=\theta^-_{\rho_G}
=
 e(G) / ( r-1)$,
 and by Lemma~\ref{lm:slope}-$(1)$
 the leading diagram $\rho_G$ must lie on 
$\widehat{\Sigma}_n(G)$.  
 \begin{enumerate}[a)]
   \item 
     For any $j=1,\dots,\ell$, by
     \eqref{fjklf} 
 we have
\begin{eqnarray*} 
 \frac{\lambda^{v(\rho_{0,G})}c_\lambda^{e(\rho_{0,G})}}{\lambda^{v(\rho_{j,G})}c_\lambda^{e(\rho_{j,G})}} & = & 
 \frac{\lambda^{(r-1)n+1}c_\lambda^{ne(G)}}{\lambda^{nr-x_j}c_\lambda^{ne(G)-y_j}}
 \\
  & = & 
 \frac{\lambda^{nr-x_0}c_\lambda^{ne(G)-y_0}}{\lambda^{nr-x_j}c_\lambda^{ne(G)-y_j}}
 \\
  & = & 
 \lambda^{x_j-x_0} c_\lambda^{y_j-y_0}
 \\
  & = & 
 \lambda^{x_j-x_0} c_\lambda^{\theta^+_G (x_j-x_0)}
 \\
  & = & 
\big(\lambda c_\lambda^{e(G) / ( r-1 )}\big)^{x_j-x_0},
\end{eqnarray*} 
    hence if $\lambda c_\lambda^{e(G)/(r-1)}\gg1$,
    we find 
$$
 \frac{\lambda^{v(\rho_{0,G})}c_\lambda^{e(\rho_{0,G})}}{\lambda^{v(\rho_{j,G})}c_\lambda^{e(\rho_{j,G})}}\gg1, 
  $$
 therefore any $\rho_G$ such that 
       $(v(\rho_G),e(\rho_G))=(v(\rho_{0,G}),e(\rho_{0,G}))=((r-1)n+1,ne(G))$
 is a leading diagram, 
 and this yields \eqref{r1} by \eqref{asy-2}.  
\item
 For any $j=0,\dots,\ell-1$, by \eqref{fjklf} 
     we have
\begin{eqnarray*} 
  \frac{\lambda^{v(\rho_{\ell , G})}c_\lambda^{e(\rho_{\ell , G})}}{\lambda^{v(\rho_{j , G})}c_\lambda^{e(\rho_{j , G})}}
  & = &
  \frac{\lambda^{r}c_\lambda^{e(G)}}{\lambda^{nr-x_j}c_\lambda^{ne(G)-y_j}}
  \\
  & = &
  \frac{\lambda^{nr-x_\ell}c_\lambda^{ne(G)-y_\ell}}{\lambda^{nr-x_j}c_\lambda^{ne(G)-y_j}}
  \\
  & = &
  \lambda^{x_j-x_\ell}c_\lambda^{y_j-y_\ell}
  \\
  & = &
  \lambda^{x_j-x_\ell}c_\lambda^{\theta^+_G (x_j-x_\ell )}
  \\
  & = &
  \big(\lambda c_\lambda^{e(G) / ( r-1) }\big)^{x_j-x_\ell}, 
  \end{eqnarray*}
 hence if $\lambda c_\lambda^{e(G)/(r-1)}\ll1$,
    we find 
$$
     \frac{\lambda^{v(\rho_{\ell , G})}c_\lambda^{e(\rho_{\ell , G})}}{\lambda^{v(\rho_{j , G})}c_\lambda^{e(\rho_{j , G})}}\gg 1, 
  $$
 therefore any $\rho_G$ such that 
 $(v(\rho_G),e(\rho_G))=(v(\rho_{\ell , G}),e(\rho_{\ell , G}))=(r,e(G))$
 is a leading diagram, 
 and this yields \eqref{r2} by \eqref{asy-2}.  
 \item
 If $\lambda c_\lambda^{e(G)/(r-1)}\asymp 1$
  then any diagram $\rho_{i,G}$, $i=0,1, \ldots , \ell$
 on the segment $\widehat{\Sigma}_n(G)$
 is a leading diagram by Lemma~\ref{lm:slope}-$(2)$. 
 Furthermore, by choosing $j=\ell$ with
 $(v(\rho_G),e(\rho_G))=(v(\rho_{\ell , G}),e(\rho_{\ell , G}))=(r,e(G))$, 
 we have 
$$
  \kappa_n(N_G)=\Theta\big(\lambda^r c_\lambda^{e(G)}\big)
  =\lambda \Theta\big(\lambda^{r-1} c_\lambda^{e(G)}\big)
  \asymp \lambda, 
$$ 
  which yields \eqref{r3} by \eqref{asy-2}.
\end{enumerate} 
~% \begin{eqnarray}
%   \frac{\lambda^{(r-1)n+1}c_\lambda^{ne(G)}}{\lambda^{r}c_\lambda^{e(G)}}%=\frac{\lambda^{nr-x_0}c_\lambda^{ne(G)-y_0}}{\lambda^{nr-x_\ell}c_\lambda^{ne(G)-y_\ell}}
%   =\lambda^{(n-1)(r-1)}c_\lambda^{(n-1)e(G)}=\left(\lambda c_\lambda^{\frac{e(G)}{r-1}}\right)^{(n-1)(r-1)}\gg1.
% \end{eqnarray}
% Also, 
\end{Proof}
Combining Lemmas~\ref{lma:stronglybal} and
 \ref{lm:slope} shows that as long as $G$ is strongly balanced, we can say that the leading asymptotic order is fully determined by either the maximal connected partition or the completely overlap one. 
\begin{remark}
  Combining Lemma~\ref{lma:stronglybal} and Proposition~\ref{dich1}, we know that for a connected and strongly balanced graph $G$ with $r=v(G)$,
  if $\lambda c_\lambda^{e(G) / ( r-1)}\gg1$, i.e. $c_\lambda\gg\lambda^{- ( r-1 )/e(G)}$, 
  then
  $$\kappa_n(N_G)=\Theta\big(\lambda^{(r-1)n+1}c_\lambda^{ne(G)}\big).
  $$
  The claim in \cite[Proposition~6.5]{LiuPrivault} only holds for $G$ strongly balanced and $c_\lambda\gg\lambda^{-\frac{r-1}e(G)}$. Note that except when
   $G$ is a tree we have $(r-1)/e(G)<1$, which means the restraint $c_\lambda\gg\lambda^{- (r-1) / e(G)}$ is stricter than the ``Dilute regime'' in most cases. 
\end{remark}
 This strongly balanced condition turns out to be necessary for this dichotomy phenomenon, as shown in the following counterexample. 
\begin{example}
  Consider the connected graph $G$ of Figure~\ref{fig:diagram5}
  which not strongly balanced, with $v(G)=e(G)=4$. 
\begin{figure}[H]
\captionsetup[subfigure]{font=footnotesize}
\centering
\begin{tikzpicture}
\filldraw [blue] (1,0) circle (1.5pt);
\filldraw [blue] (1,2) circle (1.5pt);
\filldraw [blue] (2.5,1) circle (1.5pt);
\filldraw [blue] (4,1) circle (1.5pt);
\draw[thick] (1,0) -- (1,2) -- (2.5,1) -- (1,0);
\draw[thick] (2.5,1) -- (4,1);
\end{tikzpicture}%
\caption{An example of a not strongly balanced graph.}\label{fig:diagram5} 
\end{figure}

\vskip-0.5cm

  \noindent
  From Figure~\ref{fig:diagram6}, we can see that there are some other leading diagrams on
 $\widehat{\Sigma}_2(G)$, which indicates that the asymptotic order of $\kappa_2(G)$ (and also $\kappa_3(G)$) cannot be fully determined by the so-called maximal connected partition and ``completely overlap'' partition.  

\begin{figure}[H]
\begin{subfigure}{.5\textwidth}
  \vskip1.5cm
  \begin{tikzpicture}
\draw[->] (0,0) -- (0,3);
\node[anchor=east,font=\tiny] at (-0.05,3) {y};
\draw[->] (0,0) -- (5,0);
\node[anchor=north,font=\tiny] at (5,-0.05) {x};
\draw[thick] (-0.05,0.5) -- (0.05,0.5);
\node[anchor=east,font=\tiny] at (-0.05,0.5) {1};
\draw[thick] (-0.05,1) -- (0.05,1);
\node[anchor=east,font=\tiny] at (-0.05,1) {2};
\draw[thick] (-0.05,2) -- (0.05,2);
\node[anchor=east,font=\tiny] at (-0.05,2) {4};
\draw[thick] (1,-0.05) -- (1,0.05);
\node[anchor=north,font=\tiny] at (1,-0.05) {1};
\draw[thick] (2,-0.05) -- (2,0.05);
\node[anchor=north,font=\tiny] at (2,-0.05) {2};
\draw[thick] (3,-0.05) -- (3,0.05);
\node[anchor=north,font=\tiny] at (3,-0.05) {3};
\draw[thick] (4,-0.05) -- (4,0.05);
\node[anchor=north,font=\tiny] at (4,-0.05) {4};
\filldraw [blue] (4,1) circle (1.5pt);
\filldraw [blue] (3,0.5) circle (1.5pt);
\filldraw [blue] (3,0) circle (1.5pt);
\filldraw [blue] (3,1) circle (1.5pt);
\filldraw [blue] (2,0) circle (1.5pt);
\filldraw [blue] (2,0.5) circle (1.5pt);
\filldraw [blue] (1,0) circle (1.5pt);
\filldraw [blue] (4,1.5) circle (1.5pt);
\filldraw [blue] (4,2) circle (1.5pt);
\filldraw [blue] (3,1.5) circle (1.5pt);
\draw[gray, thin] (1,0) -- (4,2);
\draw[red,thick] (1,0) -- (3,1.5) -- (4,2);
\end{tikzpicture}
    \caption{Plot of $\Sigma_2(G)$ with $\widehat{\Sigma}_2(G)$ in red.}
    \label{fig:diagram6}
\end{subfigure}
\begin{subfigure}{.5\textwidth}
\begin{tikzpicture}
\draw[->] (0,0) -- (0,4.5);
\node[anchor=east,font=\tiny] at (-0.05,4.5) {y};
\draw[->] (0,0) -- (4.5,0);
\node[anchor=north,font=\tiny] at (4.5,-0.05) {x};
\draw[thick] (-0.05,0.5) -- (0.05,0.5);
\node[anchor=east,font=\tiny] at (-0.05,0.5) {1};
\draw[thick] (-0.05,1) -- (0.05,1);
\node[anchor=east,font=\tiny] at (-0.05,1) {2};
\draw[thick] (-0.05,1.5) -- (0.05,1.5);
\node[anchor=east,font=\tiny] at (-0.05,1.5) {3};
\draw[thick] (-0.05,2) -- (0.05,2);
\node[anchor=east,font=\tiny] at (-0.05,2) {4};
\draw[thick] (-0.05,2.5) -- (0.05,2.5);
\node[anchor=east,font=\tiny] at (-0.05,2.5) {5};
\draw[thick] (-0.05,3) -- (0.05,3);
\node[anchor=east,font=\tiny] at (-0.05,3) {6};
\draw[thick] (-0.05,4) -- (0.05,4);
\node[anchor=east,font=\tiny] at (-0.05,4) {8};
\draw[thick] (1,-0.05) -- (1,0.05);
\node[anchor=north,font=\tiny] at (1,-0.05) {2};
\draw[thick] (2,-0.05) -- (2,0.05);
\node[anchor=north,font=\tiny] at (2,-0.05) {4};
\draw[thick] (3,-0.05) -- (3,0.05);
\node[anchor=north,font=\tiny] at (3,-0.05) {6};
\draw[thick] (4,-0.05) -- (4,0.05);
\node[anchor=north,font=\tiny] at (4,-0.05) {8};
\filldraw [blue] (4,3) circle (1.5pt);
\filldraw [blue] (3.5,1.5) circle (1.5pt);
\filldraw [blue] (3.5,2) circle (1.5pt);
\filldraw [blue] (3,1) circle (1.5pt);
\filldraw [blue] (3,1.5) circle (1.5pt);
\filldraw [blue] (2.5,1) circle (1.5pt);
\filldraw [blue] (3.5,2.5) circle (1.5pt);
\filldraw [blue] (2.5,0.5) circle (1.5pt);
\filldraw [blue] (3.5,1) circle (1.5pt);
\filldraw [blue] (3,0.5) circle (1.5pt);
\filldraw [blue] (2,0.5) circle (1.5pt);
\filldraw [blue] (3,2) circle (1.5pt);
\filldraw [blue] (2.5,0) circle (1.5pt);
\filldraw [blue] (3,0) circle (1.5pt);
\filldraw [blue] (2,0) circle (1.5pt);
\filldraw [blue] (3.5,3) circle (1.5pt);
\filldraw [blue] (2.5,1.5) circle (1.5pt);
\filldraw [blue] (2,1) circle (1.5pt);
\filldraw [blue] (3,2.5) circle (1.5pt);
\filldraw [blue] (1.5,0) circle (1.5pt);
\filldraw [blue] (1.5,0.5) circle (1.5pt);
\filldraw [blue] (2.5,2) circle (1.5pt);
\filldraw [blue] (1,0) circle (1.5pt);
\filldraw [blue] (2,1.5) circle (1.5pt);
\filldraw [blue] (4,3.5) circle (1.5pt);
\filldraw [blue] (4,4) circle (1.5pt);
\filldraw [blue] (3.5,3.5) circle (1.5pt);
\filldraw [blue] (3,3) circle (1.5pt);
\draw[gray, thin] (1,0) -- (4,4);
\draw[red, thick] (1,0) -- (2,1.5) -- (3,3) -- (4,4);
\end{tikzpicture}%
    \caption{Plot of $\Sigma_3(G)$ and $\widehat{\Sigma}_3(G)$ in red.}
    \label{fig:diagram6-b}
\end{subfigure}
%    \caption{  Example of $\Sigma_n(G)$
  % on $G=(V_G,E)$ with $V_G=\{u_1, u_2,u_3,u_4,v_1,v_2\}$ 
%  with $n=2,3$.}
\caption{Set $\Sigma_n(G)$ and upper boundary of its convex hull for $n=2,3$.}
\label{fig4}
\end{figure}
\end{example}
As a consequence of Proposition~\ref{dich1}, we also obtain
a Poisson limit theorem for subgraph counts. 
In the sequel, we let
 $|\mathrm{Aut}(G)|$ denote the number of automorphisms of $G$.
\begin{thm}
  Let $G$ be a connected,
  strongly balanced graph with $r=v(G)$ vertices.
  If $\lim_{\lambda \to \infty} \lambda c_\lambda^{e(G)/r} = c>0$,
  then the subgraph count $\widehat{N}_G:=N_G/|\mathrm{Aut}(G)|$
  converges in distribution to the Poisson distribution with mean 
  $$\frac{c^r}{|\mathrm{Aut}(G)|}\int_{(\R^d)^r}\prod_{(i,j)\in E(G)}H(x_i,x_j)\mu(\mathrm{d}x_1)\cdots\mu(\mathrm{d}x_r).
  $$ 
\end{thm}
\begin{Proof}
 Since $\lim_{\lambda \to \infty} \lambda c_\lambda^{e(G)/r} = c>0$
 we have $\lambda c_\lambda^{e(G)/ ( r-1 )}\ll1$, 
 hence by Proposition~\ref{dich1}-$(b)$ the leading
 diagrams $\rho$ in the expansion \eqref{cum-eq} of $\kappa_n(N_G)$
 are connected non-flat set partitions $\rho\in\Pi([n]\times[r])$
 such that $|\rho|=r$ and $e(\rho_G)=e(G)$,
 and the count of such diagrams is $|\mathrm{Aut}(G)|^{n-1}$.
 Therefore, we have 
    \begin{eqnarray*}
      \lim_{\lambda\to\infty}\kappa_n(N_G) & = &
      |\mathrm{Aut}(G)|^{n-1}
      \lim_{\lambda\to\infty}\lambda^r 
      \int_{(\R^d)^r}\prod_{(i,j)\in E(G)}
      c_\lambda H(x_i,x_j)\mu(\mathrm{d}x_1)\cdots\mu(\mathrm{d}x_r)
      \\
      &=&|\mathrm{Aut}(G)|^{n-1} c^r\int_{(\R^d)^r}\prod_{(i,j)\in E(G)}H(x_i,x_j)\mu(\mathrm{d}x_1)\cdots\mu(\mathrm{d}x_r), \qquad n \geq 1, 
    \end{eqnarray*}
    and we conclude by Theorem~6.14 in \cite{JLR11}. 
\end{Proof}
Furthermore, with the first moment method and the second moment method, i.e. 
\begin{equation}\label{firstsecm}
\frac{\E(X)^2}{\E(X^2)}\le\IP(X>0)\le \E(X).
\end{equation}
for any non-negative integer-valued random variable $X$, see \cite[Page~54]{JLR11}, we consider the thresholds for subgraph containment. 
\begin{thm}
Let the above notation and assumptions prevail. If $a<k/r$, we have the following threshold phenomenon that 
\begin{itemize}
\item $\IP(N_G>0)\to0$, if $c_\lambda\ll\lambda^{-r/k}$,
\item $\IP(N_G>0)\to1$, if $c_\lambda\gg\lambda^{-r/k}$.
\end{itemize}
\end{thm}
\begin{Proof}
First, we know that $G$ is strictly balanced, because $G$ is strongly balanced. Since $\E(X)\asymp\lambda^{r}c_\lambda^{k}$, if $c_\lambda\ll\lambda^{-r/k}$, we know that $\E(X)\to0.$ Meanwhile, we know that if $c_\lambda\gg\lambda^{-\frac{r-1}{k-a}}$, 
$$\kappa_2(N_G)\asymp\lambda^{2r-1}c_{\lambda}^{2k-a},$$
and if $c_\lambda\ll\lambda^{-\frac{r-1}{k-a}}$, 
$$\kappa_2(N_G)\asymp\lambda^{r}c_{\lambda}^{k}.$$
If $c_\lambda\gg\lambda^{-r/k}$, 
\begin{equation}
\frac{\E(X)^2}{\E(X^2)}\asymp\frac{\lambda^{2r}c_\lambda^{2k}}{\lambda^{2r}c_\lambda^{2k}+\kappa_2(N_G)}\asymp\frac1{1+(\lambda^{r}c_{\lambda}^k)^{-1}\vee(\lambda c_\lambda^{a})^{-1}}\to1.
\end{equation}
Using the first and the second method \eqref{firstsecm}, the result is rather direct.
\end{Proof}

\section{Random Graph With fixed endpoints}
\noindent
Now we consider the count $N^G_{y_1,\dots,y_m}$ of subgraph with fixed endpoints $y_1,\dots,y_m$ in the RCM $G_{H_\lambda}(\Xi_{\lambda,m})$. To be more self-contained, we first recall some notation from \cite{LiuPrivault2}.\footnote{The following highlighted part is purely imported from the existing engineering preprint.}
\textcolor{cyan}{
Given $r\geq 2$ and $m\geq 1$, we 
consider a connected graph $G=(V_G,E_G)$ with edge set $E_G$ and
vertex sequence 
$V_G=(v_1, \ldots ,v_r; e_1,\ldots , e_m)$, such that
\begin{enumerate}[i)]
\item the subgraph induced by $G$ on $\{v_1, \ldots ,v_r\}$ is connected, and 
\item $e_1, \ldots ,e_m$ are not adjacent to each other in $G$. 
\end{enumerate}
In Definition~\ref{defgraph1},
 to any graph $G$ and set partition $\rho\in\Pi ([n]\times[r])$
we associate a graph $\rho_G$ whose vertices are the blocks of $\rho$.
\begin{definition}
  \label{defgraph1}
  Given $\rho$ a partition of $[n]\times[r]$
  and $G=(V_G,E_G)$ a connected graph 
  on $V_G=(v_1, \ldots ,v_r; e_1,\ldots , e_m)$, 
  we let $\rho_G$ denote the graph 
  % from $n\times r + m$ nodes denoted respectively by $(i,j) \in [n] \times [r]$, and $(j) \in [m]$,
  constructed as follows on $[m] \cup [n]\times [r]$:
\begin{enumerate}[i)]  
\item for all $j_1, j_2\in [r]$, $j_1\not= j_2$, and $i\in [n]$, 
 an edge links $(i,j_1)$ to $(i,j_2)$
 iff $\{v_{j_1},v_{j_2}\}\in E_G$. 
\item for all $(j,k)\in [r]\times [m]$ and $i\in [n]$, an edge
 links $(k)$ to $(i,j)$ iff $\{v_j,e_k\}\in E_G$; 
\item for all $i_1,i_2\in [n]$
 and $j_1,j_2\in [r]$,
 merge any two nodes $(i_1,j_1)$ and $(i_2,j_2)$ 
 if they belong to a same block in $\rho$;  
\item eliminating any redundant edges created by the above construction.
\end{enumerate}
\end{definition}
\noindent
If $\rho\in\Pi ([n]\times[r])$
takes the form $\rho = \{ b_1,\ldots , b_{|\rho |}\}$, 
the graph $\rho_G$ forms a connected graph with
$|\rho | + m$ vertices, and we reindex the set of vertices $V_{\rho_G}$
of $\rho_G$  
as $V_{\rho_G}=[|\rho | + m ]$ according to the lexicographic order
on $\inte \times \inte$, 
followed by the remaining $m$ single-digit vertices, 
indexed as $\{|\rho |+1,\ldots , |\rho | +m\}$,
see Figure~\ref{fig:diagram1}-$b)$
in which we have $|\rho | =9$, $m = 2$, and $V_{\rho_G}=(1,\dots ,9;10, 11)$. 
\begin{definition}
    Given $m$ fixed points $y_1 , \ldots,y_m \in \R^d$,
    for a.s. $\eta$ we let $N_{y_1,\ldots , y_m}^G$ denote the count of subgraphs
   in $G_H (\eta \cup \{y_1,\ldots , y_m \} )$
   that are isomorphic to $G=(V_G,E_G)$ in the sense that 
   there exists a (random) injection
   from $V_G$ into $\eta \cup \{y_1,\ldots , y_m \}$
   which is one-to-one from $\{e_1,\ldots , e_m\}$ to $\{y_1,\ldots , y_m\}$, 
   and preserves the graph structure of $G$. 
  \end{definition}
  \noindent
   According to Definition~\ref{fjkl}, we express $N_{y_1,\ldots , y_m}^G$ as  
  \begin{equation}
  \nonumber
    N_{y_1,\ldots , y_m}^G=\sum_{(x_1, \ldots ,x_r)\in\eta^{r}}f_{y_1,\ldots , y_m} (x_1, \ldots ,x_r), 
  \end{equation}
  where the random
  function $f:(\real^d)^r \to \{0,1\}$ defined as 
  \begin{equation}
  \nonumber
  f_{y_1,\ldots , y_m} (x_1, \ldots ,x_r):=
  \prod_{
    \substack{
      1 \leq i \leq r
      \\
      1 \leq j \leq m
      \\ \{v_i,e_j\}\in E_G }
  }
  \bone_{\{y_j\leftrightarrow x_i\}} 
  \prod_{\substack{ 1 \leq k,l \leq r
      \\ \{v_\ell,v_k\}\in E_G}}\bone_{\{x_\ell\leftrightarrow x_k\}},
  \qquad
   x_1,\ldots , x_r \in \R^d, 
  \end{equation}
  is independent of the Poisson point process $\eta$,
   and $\bone_{\{x\leftrightarrow y\}}=1$ if and only if
  $x\neq y$ and $x,y\in \real^d$ are connected in the
   random-connection model 
   $G_H (\eta \cup \{y_1, \ldots ,y_m\})$.
   \begin{definition}
    For $\rho\in\Pi ([n]\times[r])$ of the form
    $\rho = \{ b_1,\ldots , b_{|\rho |}\}$ 
    and $j \in [m]$, we let 
  \begin{equation}
  \nonumber
      {\cal A}^\rho_j:=\{ k \in [ |\rho | ] \ : \ \exists (s,i)\in b_k ~\mathrm{s.t.}~
      (v_i,e_j) \in E_G % , \ s\in [n], \ i\in [r]
      \} 
  \end{equation} 
  denote the neighborhood of the vertex $(|\rho | + j)$ in $\rho_G$,
  $j=1,\ldots , m$.
   \end{definition}
   \begin{prop}
    \label{mom-cumfor}
     The moments and cumulants of $N_{y_1,\ldots , y_m}^G$ admit
     the following expressions: 
    $$
      \E_\lambda \big[\big(N_{y_1,\ldots , y_m}^G\big)^n\big]=
      \sum_{\substack{\rho\in\Pi ([n]\times[r])
          \\\rho\wedge\pi=\widehat{0}} \atop {\rm (non-flat)}}
      \lambda^{|\rho |}
      \int_{(\R^d)^{|\rho|}}\prod_{\substack{ % 1 \leq i \leq |\rho| \\
          1 \leq j \leq m
          \\ i\in {\cal A}^\rho_j}}
        H(x_i,y_j)
        \ \prod_{
          \substack{1 \leq k , \ell \le|\rho|
            \\
            \{ k , \ell \}\in E_{\rho_G} 
        }}H(x_\ell,x_k)\mu ( \mathrm{d}x_1 ) \cdots \mu ( \mathrm{d}x_{|\rho|}),
        $$
        and
        \begin{equation}
          \label{cumulant-diagram1}
        \kappa_n\big(N_{y_1,\ldots , y_m}^G\big)=
        \sum_{\substack{\rho\in\Pi_{\widehat{1}} ([n]\times[r])
            \\\rho\wedge\pi=\widehat{0}} \atop {\rm (non-flat \ \! connected)}}
      \lambda^{|\rho |}
      \int_{(\R^d)^{|\rho|}}\prod_{
        \substack{%       1 \leq i \leq |\rho| \\
        1 \leq j \leq m
          \\
        i\in {\cal A}^\rho_j}}  H(x_i,y_j)
      \ \prod_{\substack{
          1\leq k , \ell \le|\rho|
          \\
          \{ k , \ell \}\in E_{\rho_G} }}H(x_\ell,x_k)\mu ( \mathrm{d}x_1)
      \cdots \mu ( \mathrm{d}x_{|\rho|} ).
    \end{equation} 
    \end{prop}
}
In the sequel we make the following assumption on measurable and symmetric 
 connection function $H:\R^d\times\R^d\to[0,1]$,
  where we let $H(x):=H(0,x)$, $x\in \real^d$. 
 \begin{assumption}
  \label{measurability1} 
 We assume that $H$ is translation invariant, i.e. $H(x,y)=H(0,y-x)$ for all $x,y\in\R^d$, and that 
\begin{equation}
  \label{integrability-1}
0<\kappa_H:=\int_{\R^d}H(x)\mathrm{d}x<\infty.
\end{equation}
\end{assumption}
It has been shown in \cite{LiuPrivault23} that the Rayleigh connection function $H(x,y):=e^{-\beta\|x-y\|^2}$, for some $\beta>0$, satisfies Assumption~\ref{measurability1}. 
 In this section, we replace $H(x,y)$ with
$$
H_\lambda (x,y) : = c_\lambda H(x,y),
\qquad x, y \in \real^d,
$$
and investigate the asymptotic behaviour of the cumulants
$\kappa_n\big(N_{y_1,\ldots , y_m}^G\big)$ in \eqref{cumulant-diagram1} when $c_\lambda\to0$ as the intensity $\lambda$ tends
to infinity. Therefore, replacing $H$ in \eqref{cumulant-diagram1} with $H_\lambda$, we have
\begin{equation}\label{cumulant-diagram2}
  \kappa_n\big(N_{y_1,\ldots , y_m}^G\big)=
        \sum_{\substack{\rho\in\Pi_{\widehat{1}} ([n]\times[r])
            \\\rho\wedge\pi=\widehat{0}} \atop {\rm (non-flat \ \! connected)}}F(\rho),
\end{equation}
with 
\begin{equation}\label{}
  F(\rho):=\lambda^{v(\rho_G)-m}c_\lambda^{e(\rho_G)}
  \int_{(\R^d)^{|\rho|}}\prod_{
    \substack{%       1 \leq i \leq |\rho| \\
    1 \leq j \leq m
      \\
    i\in {\cal A}^\rho_j}}  H(x_i,y_j)
  \ \prod_{\substack{
      1\leq k , \ell \le|\rho|
      \\
      \{ k , \ell \}\in E_{\rho_G} }}H(x_\ell,x_k)\mu ( \mathrm{d}x_1)
  \cdots \mu ( \mathrm{d}x_{|\rho|} ).
\end{equation}
For any $i=1, \ldots ,r$, let 
\begin{equation}\label{neighborhood1}
  A_i :=\{j\in [m] \ : \ \{v_i,e_j\}\in E_G\},
\end{equation}
and we also denote 
\begin{equation}
  a:=\max\{|A_i|:i=1,\dots,r\}.
\end{equation}
In analogue to Definition~\ref{xyplot}, we define $\Sigma_n(G)$ for graph with endpoints as follows, with abuse of notation. 
\begin{definition}\label{xyplot2}
  For $n\ge2$, we denote 
  \begin{equation*}
        \Sigma_n(G):=\{(nr+m-v(\rho_G),ne(G)-e(\rho_G)) \ : \ \rho_G\in
            {\rm CNF}_G(n,r)
            \} \subseteq \inte\times \inte.
    \end{equation*}
\end{definition}
We also note that the upper boundary $\widehat{\Sigma}_n(G)$ of the convex hull of $\Sigma_n(G)$ starts at $(n-1,(n-1)a)$ and ends at $((n-1)r,(n-1)e(G))$. And we say that a diagram $\rho_G\in {\rm CNF}_G(n,r)$ lies on the upper boundary $\widehat{\Sigma}_n(G)$ if the point $(nv(G)-v(\rho_G),ne(G)-e(\rho_G))$ does. A diagram $\rho_G\in{\rm CNF}_G(n,r)$ is said to be a leading diagram if 
$$\kappa_n\big(N_{y_1,\ldots , y_m}^G\big)=\Theta\big(\lambda^{v(\rho_G)-m}c_\lambda^{e(\rho_G)}\big).$$
With a similar argument as in Lemma~\ref{lm:slope}, we can show that a diagram $\rho_G\in{\rm CNF}_G(n,r)$ is a leading diagram for some $c_\lambda$ if and only if it lies on the upper boundary $\widehat{\Sigma}_n(G)$. 

\begin{prop}\label{lm:slope2}
  Let $G$ be a connected graph with $r+m$ vertices, $r\ge2$ and $m\ge0$. Assume that $\lim_{\lambda \to \infty} c_\lambda =0$, and further suppse that the upper boundary $\widehat{\Sigma}_n(G)$ is a line segment between $(n-1,(n-1)a)$ and $((n-1)r,(n-1)e(G))$. 
  \begin{enumerate}[a)]
    \item If $\lambda c_\lambda^{[e(G)-a]/ ( r-1 )}\gg1$, we have
  \begin{equation} 
    \kappa_n\big(N_{y_1,\ldots , y_m}^G\big)=\Theta\big(\lambda^{(r-1)n+1}c_\lambda^{ne(G)-(n-1)a}\big). 
  \end{equation}
  \item If $\lambda c_\lambda^{[e(G)-a]/ ( r-1 )}\ll1$, we find 
  \begin{equation}
    \kappa_n\big(N_{y_1,\ldots , y_m}^G\big)=\Theta\big(\lambda^{r}c_\lambda^{e(G)}\big).
  \end{equation}
  \item If $\lambda c_\lambda^{[e(G)-a]/ ( r-1 )}\asymp 1$, we have
  \begin{equation}
    \kappa_n\big(N_{y_1,\ldots , y_m}^G\big)=\Theta\big(\lambda c_{\lambda}^a\big).  
  \end{equation}
  \end{enumerate} 
\end{prop}
\begin{Proof}
The proof is in the same spirit as the proof of Proposition~\ref{dich1}. Without loss of generality, we write 
$$
\widehat{\Sigma}_n(G)=\{(x_0,z_0),(x_1,z_1),\dots,(x_\ell,z_\ell)\},
$$
with $x_0=n-1<x_1<\cdots<x_\ell=(n-1)r.$ 
According to Definition~\ref{xyplot2}, we can find corresponding $\rho_{i,G}\in{\rm CNF}_G(n,r)$ such that 
$$
v(\rho_{G,i})=nr+m-x_i,~~~e(\rho_{G,i})=ne(G)-z_i.
$$
Also, we write the (connected non-flat) set partition asscociated with $\rho_{i,G}$ as $\rho_i$, and recalling from \eqref{cumulant-diagram2}, we can see that each $\rho_i$ contributes 
\begin{equation}
  F(\rho_i)=\Theta\big(\lambda^{v(\rho_{i,G})-m}c_\lambda^{e(\rho_{i,G})}\big)=\Theta\big(\lambda^{nr-x_i}c_\lambda^{ne(G)-z_i}\big).
\end{equation}
Because the upper boundary is a line segment with endpoints $(n-1,(n-1)a)$ and $((n-1)r,(n-1)e(G))$, the slope of this line segment is $[e(G)-a]/(r-1)$. Hence, for any $0\le i<j\le \ell$, $x_i<x_j$, i.e. $v(\rho_{i,G})>v(\rho_{j,G})$
\begin{eqnarray*}
  \frac{F(\rho_i)}{F(\rho_j)}&\asymp&\frac{\lambda^{v(\rho_{i,G})-m}c_\lambda^{e(\rho_{i,G})}}{\lambda^{v(\rho_{j,G})-m}c_\lambda^{e(\rho_{j,G})}}\\
  &=&\lambda^{v(\rho_{i,G})-v(\rho_{j,G})}c_\lambda^{e(\rho_{i,G})-e(\rho_{j,G})}\\
  &=&\left(\lambda c_\lambda^{[e(\rho_{i,G})-e(\rho_{j,G})]/[v(\rho_{i,G})-v(\rho_{j,G})]}\right)^{v(\rho_{i,G})-v(\rho_{j,G})}\\
  &=&\left(\lambda c_\lambda^{[e(G)-a]/(r-1)}\right)^{v(\rho_{i,G})-v(\rho_{j,G})}.
\end{eqnarray*}
Therefore, when $\lambda c_\lambda^{[e(G)-a]/ ( r-1 )}\gg1$, $F(\rho_0)\gg F(\rho_j)$ for $j=1,\dots,\ell$. Due to the fact that the leading diagram has to be one that lies on the upper boundary, we further have
\begin{equation}
  \kappa_n\big(N_{y_1,\ldots , y_m}^G\big)=\Theta\big(\lambda^{v(\rho_{0,G})-m}c_\lambda^{e(\rho_{0,G})}\big)=\Theta\big(\lambda^{nr-x_0}c_\lambda^{ne(G)-z_0}\big)=\Theta\big(\lambda^{(r-1)n+1}c_\lambda^{ne(G)-(n-1)a}\big).
\end{equation}
When $\lambda c_\lambda^{[e(G)-a]/ ( r-1 )}\ll1$, $F(\rho_j)\ll F(\rho_\ell)$ for $j=0,1,\dots, \ell-1$, and we further have
\begin{equation}
  \kappa_n\big(N_{y_1,\ldots , y_m}^G\big)=\Theta\big(\lambda^{v(\rho_{\ell,G})-m}c_\lambda^{e(\rho_{\ell,G})}\big)=\Theta\big(\lambda^{nr-x_\ell}c_\lambda^{ne(G)-z_\ell}\big)=\Theta\big(\lambda^{r}c_\lambda^{e(G)}\big).
\end{equation}
Moreover, when $\lambda c_\lambda^{[e(G)-a]/ ( r-1 )}\asymp1$, for any $0\le i<j\le \ell$, $F(\rho_i)\asymp F(\rho_j)$, and 
\begin{equation}
  \kappa_n\big(N_{y_1,\ldots , y_m}^G\big)=\Theta\big(\lambda^{r}c_\lambda^{e(G)}\big)=\Theta\big(\lambda c_\lambda^{a}\big).
\end{equation}
\end{Proof}

% \begin{remark}
%   When $m=0$, we can see through Proposition~\ref{dich1} the requirement that $G$ is strongly balanced and connected ensures the upper boundary $\widehat{\Sigma}_n(G)$ is a line segment.
% \end{remark}

A $k$-hop $P_k$ is a path with $k$ edges. Due to its simple structure, $k$-hop counts has wide applications in wireless communication, c.f. \cite{privault19}. We can see that the $k$-hop count $N_{y}^{P_k}$, with one endpoint, and $N_{y_1,y_2}^{P_k}$, with two endpoints, both satisfy that the condition in Proposition~\ref{lm:slope2}, i.e. the upper boundary is a line segment. 

\begin{example}
  Let $N_{y}^{P_k}$ be the count of $k$-hop with one fixed endpoint $y\in\R^d$. We have 
  \begin{eqnarray}
    \kappa_n\big(N_{y}^{P_k}\big)=\left\{\begin{array}{ll}
      \Theta\big((\lambda c_\lambda)^{(k-1)n+1}\big) & \ \mbox{if }\ \lambda c_\lambda\gg1,
      \medskip
      \\
      \displaystyle
\Theta\big(\lambda^{k}c_\lambda^{k}\big) & \ \mbox{if }\ \lambda c_\lambda\ll1,
    \medskip
    \\
    \displaystyle
    \Theta(1) & \ \mbox{if }\ \lambda c_\lambda\asymp1.
    \end{array}
    \right.
  \end{eqnarray}
\end{example}
\begin{example}
  Let $N_{y_1,y_2}^{P_k}$ be the count of $k$-hop with two fixed endpoint $y_1,y_2\in\R^d$. We have 
  \begin{eqnarray}
    \kappa_n\big(N_{y_1,y_2}^{P_k}\big)=\left\{\begin{array}{ll}
      \Theta\big(\lambda^{(k-2)n+1}c_\lambda^{n(k-1)+1}\big) & \ \mbox{if }\ \lambda c_\lambda^{[k-1]/ ( k-2 )}\gg1,
      \medskip
      \\
      \displaystyle
\Theta\big(\lambda^{k-1}c_\lambda^{k}\big) & \ \mbox{if }\ \lambda c_\lambda^{[k-1]/ ( k-2 )}\ll1,
    \medskip
    \\
    \displaystyle
    \Theta(\lambda c_\lambda) & \ \mbox{if }\ \lambda c_\lambda^{[k-1]/ ( k-2 )}\asymp1.
    \end{array}
    \right.
  \end{eqnarray}
\end{example}

Unlike the degenerate case, i.e. when $m=0$, the upper boundary $\widehat{\Sigma}_n(G)$ in general is much more complicated when $m>0$. %\textcolor{red}{It is difficult to find a reasonable condition on the ``density'' of graph $G$ to ensure that $\widehat{\Sigma}_n(G)$ is a segment.} 

\begin{lemma}\label{lma:mbal}
  Let $G$ be a connected graph with $r+m$ vertices. If $G$ satisfies that 
  \begin{equation}\label{mbalanced}
    \frac{e(H)-a}{v(H)-m-1}\le\frac{e(G)-a}{v(G)-m-1},\qquad H\subseteq G,~v(H)>m+1,
  \end{equation} 
  then the upper boundary $\widehat{\Sigma}_n(G)$ is a line segment. 
\end{lemma}
\begin{Proof}
  To present the result in a compact form, we will first show that the requirement that the upper boundary $\widehat{\Sigma}_n(G)$ is a line segment is equivalent to 
  \begin{equation}
    \label{mbalineq}
      \frac{e(G)-a}{v(G)-m-1}\le\frac{e(\rho_G)-a}{v(\rho_G)-m-1},
      \qquad
  \rho_G\in {\rm CNF}_G(n,r).
    \end{equation}
    Because the upper boundary $\widehat{\Sigma}_n(G)$ starts at $(x_0,z_0):=(n-1,(n-1)a)$ and ends at $(x_\ell,z_\ell):=((n-1)r,(n-1)e(G))$, the requirement that the upper boundary $\widehat{\Sigma}_n(G)$ is a line segment is equivalent to that for any $(x,z)\in\Sigma_n(G)$, 
    \begin{equation}
      \frac{z_\ell-z}{x_\ell-x}\ge\frac{z_\ell-z_0}{x_\ell-x_0}=\frac{e(G)-a}{r-1}.
    \end{equation}
    Considering the Definition~\ref{xyplot2}, we obtain that the requirement itself is further equivalent to for any $\rho_G\in {\rm CNF}_G(n,r)$, 
    \begin{equation}
      \frac{e(\rho_G)-e(G)}{v(\rho_G)-m-r}\ge \frac{e(G)-a}{r-1},
    \end{equation}
    and after reorganizing, the inequality above is equivalent to \eqref{mbalineq}. It remains to show that \eqref{mbalanced} ensures \eqref{mbalineq}. Similar with the proof of Lemma~\ref{lma:stronglybal}, we apply an induction argument to see this. When $n=1$, the claim is trivial as the only element in ${\rm CNF}_G(1,r)$ is isomorphic to $G$. \textcolor{red}{
      Suppose now that \eqref{mbalineq} holds up to the rank $n\geq 1$.
  Let $\rho$ be a non-flat connected partition on $[n+1]\times[r]$
  with associated graph $\rho_G\in{\rm CNF}_G(n,r)$.
  By Lemma~\ref{restrict-partition}, up to reordering
  of $\{1,\ldots , n+1\}$ there exists 
  $\widebar{\rho}_G\in {\rm CNF}_G(n,r)$
  induced by $\rho_G$ on
     $$
     V(\widebar{\rho}_G):=
     \big\{
     b\in \rho \ : \ b \cap (\pi_1\cup \cdots \cup \pi_n) \not= \emptyset
     \big\}. 
     $$
 Let 
     $\widehat{\rho}_G\in
     {\rm CNF}_G(1,r)$
     denote the subgraph 
  induced by $\rho_G$ on
     $$
     V(\widehat{\rho}_G):=
     \big\{
     b\in \rho \ : \ b \cap \pi_{n+1} \not= \emptyset
     \big\}, 
     $$
     with
     $\widehat{\rho}_G \simeq G$ because $\rho$ is non-flat, 
     and let $H:=\widebar{\rho}_G\cap \widehat{\rho}_G$.
     Since $H\subseteq \widehat{\rho}_G$, and $v(H)\ge m+1$
     by \eqref{mbalanced} we have 
       \begin{align*}
  \frac{e(H)-a}{v(H)-m-1}\le\frac{e(G)-a}{v(G)-m-1}. 
  \end{align*} 
 Hence, by the induction hypothesis \eqref{mbalineq} applied at the rank $n$ to $\widebar{\rho}_G\in {\rm CNF}_G(n,r)$, we have 
 \begin{align*}
  \frac{e(\rho_G)-a}{v(\rho_G)-m-1}
  & =
  \frac{[e(\widebar{\rho}_G)-a]+[e(\widehat{\rho}_G)-a]-[e(H)-a]}{
    v(\widebar{\rho}_G)+v(\widehat{\rho}_G)-v(H)-m-1}
  \\
  & =
  \frac{[e(\widebar{\rho}_G)-a]+[e(G)-a]-[e(H)-a]}{v(\widebar{\rho}_G)+v(G)-v(H)-m-1}
  \\
           & \geq 
  \frac{[v(\widebar{\rho}_G)-m-1]\frac{e(G)-a}{v(G)-m-1}  
   +[e(G)-a]-[v(H)-m-1]\frac{e(G)-a}{v(G)-m-1}
   }{
    v(\widebar{\rho}_G)+v(G)-v(H)-m-1}
  \\
           & = 
  \frac{e(G)-a}{v(G)-m-1}. 
  \end{align*} 
    }
\end{Proof}
\begin{remark}\label{special-case}
  When $m=0$, the condition \eqref{mbalanced} coincides with the strongly balanced condition \eqref{strongbaldef}. When $m=1$ and $a=1$, the condition \eqref{mbalanced} is exactly the $K_2$-balanced condition \eqref{k2baldef}.
\end{remark}
\begin{remark}
  Combining Remark~\ref{graph-example} and Remark~\ref{special-case}, we can see that when $m=a=1$, the claim in Proposition~\ref{lm:slope2} applies to trees, cycles and complete graphs.
\end{remark}

% \begin{lemma}
% ~Suppose $G$ is a connected graph with $r+m$ vertices, $r\ge2$. Each point $(x,z)\in\widehat{\Sigma}_n(G)$, lying on the upper boundary, must be of the form
% \begin{equation}
%   x=i(n-1),
% \end{equation} 
% for some $1\le i\le r$. In other words, if $\rho_G\in{\rm CNF}_G(n,r)$ is a leading diagram for some $c_\lambda$, then the number of vertices must be of the form 
% \begin{equation}
%   v(\rho_G)=n(r-i)+i+m,
% \end{equation}
% for some $1\le i\le r$.
% \end{lemma}

\appendix

\section{Convex hull code}
\label{fjkldsf}
\noindent  
The following SageMath code determines the convex hull of $\Sigma_n(G)$ 
and its upper boundary $\widehat{\Sigma}_n(G)$,
see Figures~\ref{fig1}-\ref{fig2} and \ref{fig4}.

\medskip
\smallskip

%colors linux
\begin{lstlisting}
def partitions(points):
    if len(points) == 1:
        yield [ points ]
        return
    first = points[0]
    for smaller in partitions(points[1:]):
        for m, subset in enumerate(smaller):
            yield smaller[:m] + [[ first ] + subset]  + smaller[m+1:]
        yield [ [ first ] ] + smaller
def nonflat(partition, r):
    return all(len(set((x - 1) // r for x in subset)) == len(subset) for subset in partition)
def connected(partition,n,r):
    q = []; c = 0
    if n  == 1: return all([len(j)==1 for j in partition])
    for j in partition:
        jk = list(set(map(lambda x: (x-1)//r,j)))
        if(len(jk)>1):            
            if c == 0:
                q = jk; c += 1
            elif(set(q) & set(jk)):
                d=[y for y in (q+jk) if y not in q]
                q = q + d
    return n == len(set(q))
def connectednonflat(n,r):
    points = list(range(1,n*r+1))
    randd = []
    for m, p in enumerate(partitions(points), 1):
        randd.append(sorted(p))
    for rou in range(r,(r-1)*n+2):    
        rs = [d for d in randd if (nonflat(d,r) and len(d)==rou)]
        rss = [e for e in rs if connected(e,n,r)]
        print("Connected non-flat partitions with",rou,"blocks:",len(rss))
    cnfp = [e for e in randd if (connected(e,n,r) and nonflat(e,r))]
    print("Connected non-flat set partitions:",len(cnfp))
    return cnfp
def graphs(G,setpartition,n):
    r=len(set(flatten(G)));rhoG = []
    for j in range(n):
        for hop in G: rhoG.append([r*j+hop[0],r*j+hop[1]])
    for i in setpartition:
        if(len(i)>1):
            b = []
            for j in rhoG:
                b.append([i[0] if ele in i else ele for ele in j])
            rhoG = b
    for i in rhoG: i.sort()
    return rhoG
def convexhull(n,G):
    r=len(set(flatten(G)));
    x,y=var("x,y")
    cnfp=connectednonflat(n,r)
    L=[]
    for setpartition in cnfp: 
        rhoG=graphs(G,setpartition,n)
        edgesrhoG = [i for n, i in enumerate(rhoG) if i not in rhoG[:n]]
        vertrhoG = set(flatten(edgesrhoG));
        L.append((n*r-len(vertrhoG),n*len(G)-len(edgesrhoG)))
    return sorted(set(L))
\end{lstlisting}

\footnotesize
\bibliographystyle{apalike}
\bibliography{ref}
% \begin{thebibliography}{apalike}

% \bibliographystyle{plainnat}

% \bibitem[BR12]{balakrishnan}
% R.~Balakrishnan and K.~Ranganathan.
% \newblock {\em A textbook of graph theory}.
% \newblock Universitext. Springer, New York, second edition, 2012.

% \bibitem[JLR11]{JLR} 
% S.~Janson, T.~ \L uczak, and Ruci\'{n}ski. 
% \newblock {\em Random graphs}, John Wiley \& Sons, 2011.

% \bibitem[LP23a]{LiuPrivault}
% Q.~Liu and N.~Privault.
% \newblock Normal approximation of subgraph counts in the random-connection
%   model.
% \newblock Preprint arXiv:2301.12145, 32 pages, 2023.

% \bibitem[LP23b]{LiuPrivault2}
% Q.~Liu and N.~Privault.
% \newblock Subgraph counting with fixed endpoints in the random-connection model.
% \newblock in preparion, 2023.

% \bibitem[LR92]{luczakrucinski92} 
% T.~\L uczak, and A Ruci\'nski. 
% \newblock Convex hulls of dense balanced graphs. 
% \newblock {\em J. Comput. Appl. Math.}, 41: 205-213, 1992.

% \bibitem[Pri19]{prkhp}
% N.~Privault.
% \newblock Moments of $k$-hop counts in the random-connection model.
% \newblock {\em J. Appl. Probab.}, 56(4):1106--1121, 2019.

% \bibitem[Ruc88]{ruci88}
% A.~Ruci\'nski.
% \newblock When are small subgraphs of a random graph normally distributed?. 
% \newblock{\em Probab. Theory Related Fields}, 78(1), 1--10, 1988.
% \end{thebibliography}
\end{document}

